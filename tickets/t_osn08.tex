\subsection*{OSN 8 Степенные ряды в действительной и комплексной области. Радиус сходимости.}

\textbf{Степенной ряд и область его сходимости.}

\textbf{Степенным рядом} называется функциональный ряд вида
$$ a_0 + \displaystyle\sum_{n=1}^{\infty}a_n x^n =a_0 +a_1 x+a_2 x^2 +\dots,$$
где $a_0, a_1, a_2,\dots, a_n,\dots$ --- постоянные вещественные числа, называемые \textbf{коэффициентами ряда}.

Составим с помощью коэффициентов $a_n$ ряда числовую последовательность:
\begin{equation}
    \{\sqrt[n]{|a_n|}\},~(n = 1,2,\dots)
    \label{cond}
\end{equation}

\textbf{Теорема Коши-Адамара.}
\begin{enumerate}
    \item Если последовательность \ref{cond} не ограничена, то степенной ряд сходится лишь при $x = 0$.
    \item Если последовательность \ref{cond} ограничена и имеет верхний предел $L > 0$, то ряд абсолютно сходится для значений $x$, удовлетворяющих $|x| < \frac{1}{L}$, и расходится для значений $x$, удовлетворяющих неравенству $|x| > \frac{1}{L}$.
    \item Если последовательность \ref{cond} ограничена и ее верхний предел $L = 0$, то ряд абсолютно сходится для всех значений $x$.
\end{enumerate}

\begin{proof}
\begin{enumerate}
    \item Пусть последовательность \ref{cond} не ограничена. Тогда при $x \neq 0$ последовательность $|x|\sqrt[n]{|a_n|} = \sqrt[n]{|a_n x^n|}$
    также не ограничена, т.е. у этой последовательности имеются члены со сколь угодно большими номерами $n$, удовлетворяющие неравенству $\sqrt[n]{|a_n x^n|} > 1$, или $|a_n x^n| > 1$.
    Но это означает, что для ряда (при $x\neq 0$) нарушено необходимое условие сходимости, т. е. ряд расходится при $x\neq 0$.
    \item Пусть последовательность \ref{cond} ограничена и ее верхний предел $L > 0$. Докажем, что ряд абсолютно сходится при $|x| < \frac{1}{L}$ , и расходится при $|x| > \frac{1}{L}$.
    \begin{itemize}
        \item Фиксируем сначала любое $x$, удовлетворяющее неравенству $|x| < \frac{1}{L}$. Тогда найдется $\varepsilon > 0$, такое, что $|x| < \frac{1}{L+\varepsilon}$.

        В силу свойств верхнего предела все элементы $\sqrt[n]{|a_n|}$, начиная с некоторого номера $n$, удовлетворяют неравенству $\sqrt[n]{|a_n|} < L + \frac{\varepsilon}{2}$.

        Таким образом, начиная с этого номера $n$, справедливо неравенство $\sqrt[n]{|a_n x^n|} = |x|\sqrt[n]{|a_n|} < \frac{L + \frac{\varepsilon}{2}}{L+\varepsilon} < 1$, т. е. ряд абсолютно сходится по признаку Коши.
        \item Фиксируем теперь любое $x$, удовлетворяющее неравенству $|x| > \frac{1}{L}$. Тогда найдется $\varepsilon > 0$ такое, что $|x| > \frac{1}{L - \varepsilon}$.

        По определению верхнего предела из последовательности \ref{cond} можно выделить подпоследовательность $\{\sqrt[n_k]{|a_{n_k}|}\},~(k=1,2,\dots)$, сходящуюся к $L$.
        Но это означает, что, начиная с некоторого номера $k$, справедливо неравенство $L-\varepsilon< \sqrt[n_k]{|a_{n_k}|} <L+\varepsilon$.

        Таким образом, начиная с этого номера $k$, справедливо неравенство $\sqrt[n_k]{|a_{n_k}x^{n_k}|}=|x|\sqrt[n_k]{|a_{n_k}|}>\frac{L-\varepsilon}{L-\varepsilon}=1$, или $|a_{n_k}x^{n_k}| > 1$, откуда видно, что нарушено необходимое условие сходимости ряда и он расходится.
    \end{itemize}
    \item Пусть последовательность \ref{cond} ограничена и ее верхний предел $L = 0$.
    Докажем, что ряд абсолютно сходится при любом $x$.
    Фиксируем произвольное $x\neq 0$ (при $x = 0$ ряд заведомо абсолютно сходится). Поскольку верхний предел $L = 0$ и последовательность \ref{cond} не может иметь отрицательных предельных точек, число $L = 0$ является единственной предельной точкой, а следовательно, является пределом этой последовательности, т. е. последовательность является бесконечно малой.
    Но тогда для положительного числа $\frac{1}{2|x|}$ найдется номер, начиная с которого $\sqrt[n]{|a_n|} < \frac{1}{2|x|}$.
    Стало быть, начиная с указанного номера, $\sqrt[n]{|a_n x^n|}=|x^n|\sqrt[n]{|a_n|}<\frac{1}{2}<1$, т. е. ряд абсолютно сходится к признаку Коши.
\end{enumerate}
\end{proof}


\textbf{Радиус сходимости.}

\textbf{Теорема.} Для каждого степенного ряда, если он не является рядом, сходящимся лишь в точке $x = 0$, $\exists R > 0$ (возможно, равное бесконечности) такое, что этот ряд абсолютно сходится при $|x| < R$ и расходится при $|x| > R$.
Это число $R$ называется \textbf{радиусом сходимости} рассматриваемого степенного ряда, 
а интервал $(-R, R)$ называется \textbf{промежутком сходимости} этого ряда.
Для вычисления радиуса сходимости справедлива формула
$$R=\frac{1}{\overline{\lim\limits_{n\to\infty}}\sqrt[n]{|a_n|}}$$
(в случае, когда $\overline{\lim\limits_{n\to\infty}}\sqrt[n]{|a_n|} = 0,~R = \infty$)

\begin{proof}
Очевидно из предыдущей теоремы
\end{proof}

\textbf{Для случая комплексного пространства:}

Ряд вида $\displaystyle\sum_{n=0}^{\infty}a_n(z-z_0)^n$ называется \textbf{степенным рядом} с центром разложения в точке $z_0$, где $\{a_n\}$ --- фиксированная последовательность комплексных чисел.

\textbf{Теорема Коши-Адамара.} 

Если $R = 0$ (т. е. $\overline{\lim\limits_{n\to\infty}}\sqrt[n]{|a_n|} = \infty$), то ряд $\displaystyle\sum_{n=0}^{\infty}a_n(z-z_0)^n$ сходится только в точке $z_0$.

Если $R = \infty$ (т. е. $\overline{\lim\limits_{n\to\infty}}\sqrt[n]{|a_n|} = 0$), то ряд сходится абсолютно во всей комплексной плоскости $C$.

Если $0 < R < \infty$, то ряд сходится абсолютно внутри круга $|z-z_0| < R$, вне замкнутого круга ряд расходится.

% Можно даже привести пример (на пальцах) что сходимость неравномерная: для ряда sum(z^n) при z € R+ приближаясь к точке z=1 можно найти сколь угодно медленно сходящийся ряд

\begin{proof}
Доказательство по сути идентично доказательству для вещественного случая
\end{proof}
