\subsection{Понятие имитационной модели. Примеры средств имитационного моделирования.
Типовая архитектура средств имитационного моделирования (OmNet++, NS3).}

\subsubsection*{Понятие имитационной модели}
Имитационная модель — это алгоритмическая математическая модель, отражающая
поведение исследуемого объекта во времени при задании внешних воздействий.
Она воспроизводит процесс функционирования системы,
имитируя элементарные явления с сохранением их логической структуры и временной последовательности.

\textbf{Основные особенности:}
\begin{itemize}
    \item \textbf{Алгоритмическая основа:}
		описывает поведение объекта через алгоритмы,
		учитывающие изменение состояний во времени.
    \item \textbf{Динамическое представление:}
		моделирует процессы в реальном или условном времени,
		сохраняя логику взаимодействий.
    \item \textbf{Универсальность:}
		применима для широкого круга задач,
		но результаты зависят от конкретных входных данных.
    \item \textbf{Методы продвижения времени:}
		постоянный шаг (непрерывные/потактовые модели),
		дискретно-событийный подход, гибридные методы.
    \item \textbf{Интеграция предметной области:}
		требует поддержки специфических понятий и логики системы
		в инструментах моделирования.
\end{itemize}

\subsubsection*{Этапы построения имитационной модели}
\textbf{Основные этапы:}
\begin{itemize}
    \item \textbf{Анализ требований и проектирование:}
        \begin{itemize}
            \item Определение целей моделирования (например, оценка производительности системы, анализ загрузки ресурсов).
            \item Построение концептуальной модели:
                \begin{itemize}
                    \item Формализация структуры системы (компоненты, связи, состояния).
                    \item Описание алгоритмов функционирования (взаимодействие элементов, обработка событий).
                    \item Упрощение и абстракция (исключение несущественных деталей).
                \end{itemize}
            \item Проверка адекватности концептуальной модели (экспертная оценка, сравнение с реальным объектом).
        \end{itemize}
    
    \item \textbf{Реализация модели:}
        \begin{itemize}
            \item Выбор инструментов: язык программирования (C++, Python), библиотеки (SimpleSim), среды моделирования.
            \item Программирование компонентов:
                \begin{itemize}
                    \item Реализация логики событий (например, обработка запросов сервером).
                    \item Моделирование времени (дискретно-событийный подход, постоянный шаг).
                \end{itemize}
            \item Отладка и верификация (проверка корректности кода, соответствия концептуальной модели).
        \end{itemize}
    
    \item \textbf{Экспериментирование и анализ:}
        \begin{itemize}
            \item Планирование экспериментов:
                \begin{itemize}
                    \item Выбор варьируемых параметров (интенсивность запросов, количество серверов).
                    \item Определение метрик (загруженность сервера, длина очереди).
                \end{itemize}
            \item Прогон модели:
                \begin{itemize}
                    \item Запуск с разными наборами входных данных.
                    \item Сбор статистики (например, временные диаграммы работы системы).
                \end{itemize}
            \item Анализ результатов:
                \begin{itemize}
                    \item Интерпретация данных (выявление «узких мест», оптимизация параметров).
                    \item Формулирование выводов (рекомендации по настройке системы).
                \end{itemize}
        \end{itemize}
\end{itemize}

\textbf{Пример из лекции:} Модель сервера с клиентами.
На этапе реализации использован событийный подход с классами \texttt{Client},
\texttt{Server} и календарём событий.
В экспериментах анализировались максимальная длина очереди и загруженность сервера.

\subsubsection*{Примеры средств имитационного моделирования}
\begin{itemize}
    \item \textbf{OMNeT++}:
    \begin{itemize}
        \item Среда для дискретно-событийного моделирования с открытым исходным кодом.
        \item Использует язык \texttt{NED} для описания иерархии модулей (простые/составные) и связей между ними.
        \item Поведение компонентов реализуется на C++ через обработку сообщений (\texttt{cMessage}) и планирование событий (\texttt{scheduleAt}).
        \item Примеры применения: моделирование сетей, обработка прерываний, клиент-серверные системы.
    \end{itemize}

    \item \textbf{NS-3}:
    \begin{itemize}
        \item Фреймворк для эмуляции сетевых протоколов (уровень L3 и выше).
        \item Поддерживает узлы (\texttt{Node}), сетевые интерфейсы (\texttt{NetDevice}) и каналы (\texttt{Channel}).
        \item Сбор данных через трассировку в формате \texttt{PCAP} и журналирование.
    \end{itemize}

    \item \textbf{AnyLogic}:
    \begin{itemize}
        \item Универсальная платформа, поддерживающая агентное, системно-динамическое и дискретно-событийное моделирование.
        \item Интеграция с UML-диаграммами состояний для описания параллельных процессов.
    \end{itemize}
\end{itemize}

\subsubsection*{Типовая архитектура средств имитационного моделирования}
\begin{itemize}
    \item \textbf{Общие компоненты}:
    \begin{itemize}
        \item \textbf{Планировщик событий}: управляет очередью событий с временными метками (дискретно-событийный подход).
        \item \textbf{Моделируемые объекты}:
        \begin{itemize}
            \item В OMNeT++: модули (\texttt{cSimpleModule}) с методами \texttt{initialize()}, \texttt{handleMessage()}.
            \item В NS-3: узлы, протоколы (TCP/IP), каналы связи.
        \end{itemize}
        \item \textbf{Средства сбора данных}:
        \begin{itemize}
            \item Логирование (OMNeT++ — \texttt{.vec/.sca}, NS-3 — \texttt{PCAP}).
            \item Визуализация через графические интерфейсы (например, \texttt{Tkenv} в OMNeT++).
        \end{itemize}
    \end{itemize}

    \item \textbf{Методы организации моделей}:
    \begin{itemize}
        \item \textbf{Последовательные процессы}: реализация через потоки с методами \texttt{main()}, \texttt{signal\_event()}.
        \item \textbf{Конечные автоматы}:
        \begin{itemize}
            \item Состояния с действиями при входе/выходе (например, «Свободен», «Занят»).
            \item Параллельные состояния и синхронизация через триггеры и сторожевые условия.
        \end{itemize}
    \end{itemize}

    \item \textbf{Этапы разработки}:
    \begin{itemize}
        \item Анализ требований и проектирование концептуальной модели (диаграммы состояний).
        \item Реализация модели: выбор языка (C++ для OMNeT++, Python для NS-3), отладка.
        \item Проведение экспериментов: параметризация, анализ результатов.
    \end{itemize}
\end{itemize}

\begin{center}
    \begin{tabular}{|l|l|l|}
        \hline
        \textbf{Характеристика} & \textbf{OMNeT++} & \textbf{NS-3} \\
        \hline
        Парадигма & Дискретно-событийная + конечные автоматы & Сетевая эмуляция \\
        \hline
        Описание структуры & Язык \texttt{NED} & C++/Python \\
        \hline
        Отладка & Графический интерфейс \texttt{Tkenv} & Трассировка \texttt{PCAP} \\
        \hline
    \end{tabular}
\end{center}

\subsubsection*{Примечания}
\begin{itemize}
    \item Для моделирования прерываний в OMNeT++ используются методы \texttt{scheduleAt} и \texttt{cancelEvent}.
    \item Примеры применения: моделирование серверов FTP, сокетов, процессов в Linux.
    \item Гибридные модели сочетают дискретные события и непрерывные процессы для сложных систем.
\end{itemize}

Материалы, основываясь на которых написан билет~\cite{imitation_modeling}
