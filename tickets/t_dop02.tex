\subsection*{DOP 2 Графы,  деревья,  планарные графы;  их свойства.  Оценка числа деревьев.}

\textbf{Определения}
\begin{itemize}
\item \textbf{Графом} называется произвольное множество элементов V и произвольное семейство $E$ пар из $V$. \textit{Обозначение}: $G = (V, E)$.
Если элементы из $E$ рассматривать как неупорядоченные пары, то граф называется \textbf{неориентированным}, а пары называются \textbf{рёбрами}. 
Если же элементы из $E$ рассматривать как упорядоченные, то граф \textbf{ориентированный}, а пары --- \textbf{дуги}.

Пара вида $(a, a)$ называется \textbf{петлёй}.

Если пара $(a, b)$ встречается в семействе $E$ несколько раз, то она называется \textbf{кратным ребром} (\textit{кратной дугой}).


В дальнейшем условимся граф без петель и кратных рёбер называть простым графом; граф без петель, в котором допустимы кратне ребра --- \textit{мультиграфом}; граф, в котором допустимы петли, с петлями --- \textit{псевдографом}. В дальнейшем, если не оговаривается обратное, под графом будем подразумевать простой граф.

\item Две вершины графа называются \textbf{смежными}, если они соединены ребром. Говорят, что вершина и ребро \textbf{инцидентны}, если ребро содержит вершину. \textbf{Степенью вершины} $(deg~v)$ называется количество рёбер, инцидентных данной вершине. Для псевдографа полагают учитывать петлю дважды.

\item \textbf{Деревом} называется связный граф без циклов. % \todo{NO} не, тут все ок

\item \textbf{Остовное дерево} графа -- это дерево, подграф данного графа, с тем же числом вершин, что и у исходного графа. Неформально говоря, остовное дерево получается из исходного графа удалением максимального числа рёбер, входящих в циклы, но без нарушения связности графа. Остовное дерево включает в себя все $n$ вершин исходного графа и содержит $n-1$ ребро.


\item  Дерево, в котором одна из вершин выделена, называется \textbf{корневым деревом}. Выделенная вершина называется корнем.

\item \mathLet \ имеются корневые деревья $D_i = (V_i , E_i),~i = \overline{1,m}$ с корнями $v_i$, и $V_i$ попарно не пересекаются. 
Тогда граф $D = (V, E)$, полученный следующим образом: $V = \bigvee^m_{i=1} V_i \cup \{v\}$, где $v \notin V_i,~i = \overline{1,m},~E = \bigvee^m_{i=1} E_i \cup \{(v, v_1), \dots, (v, v_m)\}$, и в котором выделена вершина $v$, называется \textbf{корневым деревом}.

\item \textbf{Упорядоченным корневым деревом} называется корневое дерево, в котором задан порядок поддеревьев и каждое поддерево является упорядоченным поддеревом.

\item \textbf{Планарный граф} --- граф, который можно изобразить на плоскости без пересечений рёбер.
Если имеется планарная реализация графа и мы <разрежем> плоскость по всем линиям этой реализации, то плоскость распадётся на \textbf{грани} (одна из граней бесконечна, она называется внешней гранью).

\item Два графа $G_1$ и $G_2$ называются \textbf{изоморфными}, если существует биекция $f: V_1 \rightarrow V_2$ такая, что любые вершины $v$ и $w$ графа $G_1$ смежны $\Leftrightarrow$ $f(v)$ и $f(w)$ смежны в графе $G_2$.

\item Говорят, что граф $G'$ получен из графа $G$ \textbf{подразбиением ребра} $e = (v,w) \in E$, если $V' = V \cup \{u\}$, где $u \notin V$; $E' = E \setminus \{(v,w)\} \cup \{(v, u), (u,w)\}$. Граф $G'$ называется \textbf{подразбиением графа} $G$, если $G'$ может быть получен из $G$ конечным числом подразбиений ребер.

\item Графы $G_1$ и $G_2$ называются \textbf{гомеоморфными}, если найдутся изоморфные их подразбиения $G'_1$ и $G'_2$ соответственно.

\end{itemize}

\noindent\textbf{Свойства}
\begin{itemize}
\item \textbf{Утв 1.} 
\mathLet \ в графе $G = (V, E)$ p вершин и q ребер ($|V|=p, |E| = q$) и $deg V_i$ - степень вершин $V_i$, тогда
$\sum_{i = 1}^p degV_i = 2q$. \newline
\begin{proof}
Равенство следует из того, что слева каждое ребро учитывается два раза.
\end{proof}


\item \textbf{Лемма 1.} 
Если граф $G = (V, E)$ связный и ребро $(a, b)$ содержится в некотором цикле в графе $G$, то при выбрасывании из графа $G$ ребра $(a, b)$ снова получится связный граф.\newline
\begin{proof}
Если путь из $V_i$ в $V_j$ не проходил через $(a,b)$, то все останется как есть. Если выкинуть $(a,b)$ и $(a,b)$ принадлежал циклу, то все останется как есть.
\end{proof}


\item \textbf{Теорема 1.} 
Любой связный граф содержит хотя бы одно остовное дерево.\newline
\begin{proof}
$\mathLet ~ G=(V,E)$ - связный граф. Если в G нет циклов, то G -искомое остовное дерево. Если в G $\exists$ цикл, то выкинем из G $\forall$ ребро $(a,b)$ из цикла, тогда по Лемме 1 останется связный подграф. Если в нем нет циклов, то он - искомое остовное дерево. Иначе пролжаем процесс выбрасывания ребер из циклов. Процесс конечный, так как граф конечный.
\end{proof}

\item \textbf{Лемма 2.} 
Если к связному графу добавить новое ребро на тех же вершинах, то появится цикл. \newline
\begin{proof}
$\mathLet ~ (V_i, V_j) \notin E, i \neq j$. Так как граф связный, то в нем $\exists$ путь из $V_i$ в $V_j \implies \exists$ простая цепь из $V_i$ в $V_j \implies $ эта цепь + ребро $(V_j, V_i)$  - цикл в данном подграфе
\end{proof}

\item \textbf{Лемма 3.} 
\mathLet \ в графе $G = (V, E)$ $p$ вершин и $q$ рёбер. Тогда 
1) в $G$ не менее $p - q$ связных компонент. 
2) Если при этом в $G$ нет циклов, то $G$ состоит ровно из $p - q$ связных компонент.\newline
\begin{proof}
1) \faEye \ граф $G_0 = (V, \emptyset)$ и будем из $G_0$ получать граф $G = (V, E)$, добавляя по одному ребру из E. В $G_0$ p связных компонент. При добавлении одного ребра к любому графу (на тех же вершинах) число связных компонент либо не уменьшается, либо уменьшается на 1. $\implies$ при добавлении к графу q ребер число св. компонент может уменьшиться на $\leq q \implies$ в графе G будет $\geq$ p-q св. компонент. (т.к. в $G_0$ p св. компонент)\newline
2) Если в графе G нет циклов, то каждое добавленное ребро должно соединять две вершины из разных св. компонент (иначе по лемме 2 появляется цикл) $\implies$ число св. компонент уменьшается ровно на 1 $\implies$ в G ровно p-q св. компонент.
\end{proof}


\item \textbf{Теорема об эквивалентных определениях дерева.}
$\mathLet ~ G = (V, E),~|V| = p,~|E| = q$, тогда следующие утверждения попарно эквивалентны:
\begin{enumerate}
    \item $G$ --- дерево.
    \item $G$ --- граф без циклов и $q = p - 1$.
    \item $G$ --- связный граф и $q = p - 1$.
    \item $G$ --- связный граф, но при удалении любого ребра становится несвязным.
    \item $G$ --- граф без циклов, но при добавлении любого ребра в нём образуется цикл. 
\end{enumerate}
\begin{proof}
$1 \rightarrow 2$ \textit{Дано}: G - связный, без циклов.\textit{Док-ть}: G - без циклов и q = p-1. \newline
По лемме 3 s=p-q=1

$2 \rightarrow 3$ \textit{Дано}: G - без циклов, q = p-1.\textit{Док-ть}: G - связный и q = p-1. \newline
По лемме 3 s = p-q =1 $\implies$ G - связный

$3 \rightarrow 4$ \textit{Дано}: G - связный, q = p-1.\textit{Док-ть}: G - связный, но при удалении любого ребра становится несвязным. \newline
Из G удаляем $\forall$ ребро - получаем G'. Для $G' s' \geq p-q'=2$. По лемме 3 G' - несвязен

$5 \rightarrow 5$ \textit{Дано}:G - связный, но при удалении любого ребра становится несвязным.\textit{Док-ть}: G - без циклов, но при добавлении любого ребра в нём образуется цикл.\newline
\mathLet \ в G есть цикл l и l - ребро из G, G' = G-l. По лемме 1 - связный (Противоречие)

$5 \rightarrow 1$ \textit{Дано}:G - без циклов, но при добавлении любого ребра в нём образуется цикл.\textit{Док-ть}: G - связный, без циклов. \newline
Если в G найдется цикл, то удалим из G любое ребро из цикла. Останется связный граф - противоречие. Значит, G без циклов
\end{proof}

% \item \textbf{Формула Эйлера.} 
% Для любой планарной реализации связного планарного графа $G = (V, E)$ с $p$ вершинами, $q$ рёбрами и $r$ гранями выполняется равенство: $p - q + r = 2$.
% \item \textbf{Утв 2.} 
% Графы $K_5$ и $K_{3,3}$ (рисунок) не являются планарными.
% \item \textbf{Теорема Понтрягина-Куратовского.} 
% Граф $G = (V, E)$ планарен тогда и только тогда, когда в нем не найдется ни одного подграфа, гомеоморфного либо графу $K_5$, либо графу $K_{3,3}$.
\end{itemize}

\textbf{Теорема: Оценка числа деревьев.} Число упорядоченных корневых деревьев с $q$ рёбрами не превосходит $4^q$.

\begin{proof}
$\mathLet ~ (D; v_0)$ — упорядоченное корневое дерево с q ребрами. Обойдем дерево D в глубину из вершины $v_0 \in V$ по порядку его поддеревьев. При таком обходе по каждому ребру пройдем два раза: первый раз при переходе в соответствующее поддерево, второй раз при возвращении из него.
По этому обходу построим код дерева D — набор k(D) из нулей и единиц длины 2q. Сначала этот код не заполнен. При проходе по очередному ребру заполняем в коде k(D) первый незаполненный разряд по следующим правилам:\newline
1) если по ребру переходим в поддерево, то в код k(D) пишем ноль;\newline
2) если по ребру возвращаемся из поддерева, то в код k(D) пишем единицу.
Тогда различным упорядоченным корневым деревьям соответствуют разные коды.
Поэтому $\delta''(q)$ не превосходит числа наборов из нулей и единиц длины 2q, т.е.
$\delta''(q) \leq 2^{2q} = 4^q$.
\end{proof}

\textbf{Теорема: формула Эйлера.} Для любой планарной реализации связного планарного графа $G=(V,E)$ с $p$ вершинами, $q$ ребрами и $r$ гранями выполнено $p-q+r=2$.

\textbf{Теорема Понтрягина-Куратовского.} Для любой реализации планарного графа $G=(V,E)$ планарен тогда и только тогда, когда в нем не найдется ни одного подграфа, гомеоморфного $K_5$ или $K_{3,3}$.

% -------- source --------
\bigbreak
[\cite[page 69-96]{replace_me}]
