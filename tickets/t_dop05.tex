% ---------------
\newcommand\eqdef{\mathrel{\stackrel{\makebox[0pt]{\mbox{\normalfont\tiny def}}}{=}}}
% ---------------


\subsection{DOP 5 Коды Боуза-Чоудхури-Хоквингема: определение, алгоритмы кодирования и декодирования.}


\textbf{Теоретический минимум по прикладной алгебре:}\\
\textbf{Простое поле Галуа} -- поле классов вычетов по модулю простого числа.
$\mathbb{Z}/(p) \cong \mathbb{Z}_p \eqdef \mathbb{F}_p$, тут $p$ -- простое.

\textbf{Расширением простого поля} называется факторкольцо $\mathbb{F}_p[x]/(a(x))$, где
$\mathbb{F}_p[x]$ -- кольцо всех многочленов переменной $x$ со коэффициентами из поля
$\mathbb{F}_p$, а $(a(x))$ -- идеал неприводимого многочлена из кольца $\mathbb{F}_p[x]$.

\textbf{Минимальным многочленом} (ММ) элемента $\beta \in \mathbb{F}_p^n$ называется
нормированный многочлен (коэф. при старшей степени $\equiv$ 1)
$m_{\beta} (x) \in \mathbb{F}_p[x]$ наименьшей степени, для которого $\beta$ является
корнем.

\textbf{Сопряженными} элементами поля $\mathbb{F}_p^t$ называются ненулевые элементы,
имеющие общий минимальный многочлен.

\textbf{Циклотомический класс} составлен из всех сопряженных элементов.

Циклотомические классы либо совпадают, либо не пересекаются в совокупности дают
\textbf{разбиение} мультипликативной группы поля $\mathbb{F}_p^t$, или ее
\textbf{разложение на классы} над $\mathbb{F}_p$.

В поле характеристики $p$ если $\alpha$ -- корень некоторого полинома,
то и $\alpha^p$ -- его корень. Поэтому циклотомические классы можно получать
возведением в степень $p$ какого-то одного его элемента. Это совпадает с построением
орбиты отображения.

\textbf{Пример.} Пусть $t = 4$ и $\alpha$ -- примитивный элемент
поля $\mathbb{F}^4_2$. Тогда его мультипликативная группа
$$
\{ \alpha, \alpha^2, \dots, \alpha^{14}, \alpha^{15} = \alpha^0 = 1 \}
$$
разлагается над $\mathbb{F}_2$ на циклотомические классы
\begin{gather*}
    C_0 = \{\alpha^0\},
    ~C_1 = \{\alpha, \alpha^2, \alpha^4, \alpha^8 \} \, (\alpha^{16} = \alpha^1 = \alpha),\\
    C_2 = \{\alpha^3, \alpha^6, \alpha^{12}, \alpha^{24} = \alpha^9\} \, (\alpha^{18} = \alpha^3),\\
    C_3 = \{\alpha^5, \alpha^{10}\} \, (\alpha^{20} = \alpha^5),\\
    C_4 = \{\alpha^7, \alpha^{14}, \alpha^{28} = \alpha^{13}, \alpha^{26} = \alpha^{11}\} \, (\alpha^{22} = \alpha^7)
\end{gather*}

\textbf{Синдромом} $s(x)$ слова $w(x)$, принятого при передаче сообщения,
закодированного циклическим кодом, и, возможно, содержащего ошибки, называют остаток
от деления $w(x)$ на многочлен $g(x)$, порождающий код.

% \textbf{Линейным $[n,k]$ кодом} называют такой блоковый $(n,k)$-код $C$, который
% образует линейное векторное подпространство размерности $k$ координатного
% пространства $W$ всех потенциально возможных скачанных слов. Символически
% $C \le \{0, 1\}^n = W$.

\textbf{Циклическим кодом} называется такой блоковый код, который инвариантен
относительно циклических сдвигов своих кодовых слов. То есть, например, сдвиг
вправо на 1 кодового слова $[0, 0, 1]$ превратит его в $[1, 0, 0]$.\\

% Рассматривать будет непосредственно линейные циклические коды, хотя последние не
% обязательно являются упомянутыми.


\textbf{Основная часть:}\\
\textbf{Коды Боуза-Чоудхри-Хоквингема (БЧХ)} - циклические коды, исправляющие
\textit{не менее} заранее заданного числа ошибок.

\textbf{Длина кода} БЧХ определяется параметром $t$, $n = 2^t - 1$. Для бинома
$x^n - 1$ рассматривается поле $\mathbb{F}_2^t$ его \textit{разложения} с некоторым
примитивным элементом $\alpha$.

\textbf{Конструктивное расстояние} БЧХ кода рассчитывается по количеству
исправляемых ошибок $r$, $\delta = 2r + 1 < n$.

\textbf{Нулями кода} БЧХ называются степени
$\alpha, \alpha^2, \ldots, \alpha^{2r}$ примитивного элемента $\alpha$ поля
$\mathbb{F}_2^t$.

\textbf{Код БЧХ} - это циклический $[n,k,d]$-код, в котором порождающий многочлен
$g(x)$ является полиномом минимальной степени, имеющий корнями \textit{все нули
кода}.

Так как нули кода - корни $g(x)$, а многочлены всех кодовых слов циклического кода
делятся на $g(x)$, то нули кода так же корни и многочленов кодовых слов.\\

\textbf{Синдромами} $s_1, \ldots, s_{2r}$ скачанного многочлена $w(x)$ при
кодировании БЧХ-кодом с нулями $\alpha, \alpha^2, \ldots, \alpha^{2r}$ являются
$s_i = w(\alpha^i),~i = \overline{1,~2r}$.

Так как $w(x) = v(x) + e(x)$, где $v(x)$ - кодовое слово, $e(x)$ - вектор
ошибок, то\\
$\forall i = \overline{1,~2r} ~:~ s_i = w(\alpha^i) = e(\alpha^i)$.

Очевидно, если все синдромы равны нулю, то $w(x)$ и есть кодовое слово.\\

\textbf{Построение кода БЧХ.}\\
Аналогично любому циклическому коду, $[n,k]$-код БЧХ задается порождающим
многочленом $g(x)$, делящим бином $x^n - 1,~k = n - \deg g(x)$.\\

\textbf{Алгоритм построения кода БЧХ, исправляющим не менее $r$ ошибок}:
\begin{enumerate}
  \item Выбрать параметр $t$ таким образом, что $n = 2^t - 1 > 2r + 1 = \delta$.
  \item Выбрать неприводимый многочлен $a(x)$ степени $t$ для поля расширения
    $\mathbb{F}_2^t \cong \mathbb{F}_2[x]/(a(x))$.
  \item Найти циклотомические классы поля $\mathbb{F}_2^t$ над $\mathbb{F}_2$. Выбрать из них те
    классы, в которые попадают все $2r$ нулей
    $\alpha, \alpha^2, \ldots, \alpha^{2r}$ кода. Пускай таких классов $h$.

  \item Найти минимальные многочлены $g_1(x), g_2(x), \ldots, g_h(x)$ каждого
    выбранного класса.
  \item  Вычислить порождающий многочлен кода
    $g(x) = g_1(x)\cdot g_2(x)\cdot\ldots\cdot g_h(x)$.
\end{enumerate}

Важно отметить, что при повышении количества исправляемых ошибок при неизменной
длине кода, он будет становится ``хуже``, например при длине кода $n = 7,~r = 3$
получим код с 7-кратным (!) повторением.\\

\textbf{\underline{Пример:}}\\
Построим код длины $n = 15$, то есть параметр $t = 4$.
Рассмотрим поле $F = \mathbb{F}_2[x]/(a(x)) \cong \mathbb{F}_2^4$, где $a(x)$ некоторый
неприводимый многочлен четвертой степени. Тогда мультипликативной группа $F*$
относительно своего примитивного элемента $\alpha$ разобьется на 5
циклотомических класса над $\mathbb{F}_2$:
\begin{gather*}
  C_0 = \{\alpha^0\},~C_1 = \{\alpha, \alpha^2, \alpha^4, \alpha^8 \},\\
  C_2 = \{\alpha^3, \alpha^6, \alpha^{12}, \alpha^9\},~
  C_3 = \{\alpha^5, \alpha^{10}\},\\
  C_4 = \{\alpha^7, \alpha^{14}, \alpha^{13}, \alpha^{11}\}
\end{gather*}

В качестве неприводимого многочлена 4ой степени возьмем
$a(x) = x^4 + x + 1$. Понятно, что он будет минимальным многочленом для
$\alpha = x$ и всего $C_1$.\\

Допустим мы хотим исправить не менее двух ошибок, то есть
$r = 2 \Rightarrow 2r = 4$, тогда нулями кода будут
$\alpha, \alpha^2, \alpha^3, \alpha^4$. Как видно из построенных выше классов,
нули попадают в $C_1$ и $C_2$. Найдем мм-ы для них:\\
1. для $g_1(x)$ вы уже нашли и он равен $a(x)$\\
2. а для $g_2(x)$ мм конструируется по элементам класса - \\
$
  g_2(x) = (x - \alpha^3)(x - \alpha^6)(x - \alpha^9)(x - \alpha^{12})
$$
  = \{\ldots\} = x^4 + x^3 + x^2 + x + 1
$.\\

Отсюда порождающий многочлен кода есть\\
$g(x) = g_1(x) \cdot g_2(x) = x^8 + x^7 + x^6 + x^4 + 1$.\\

Получим, что $m = \dim g(x) = 8,~\Rightarrow k = n - \dim g(x) = 7$. При этом\
$d = \delta = 5$ и мы получили БЧХ $[15,7,5]$-код со скоростью $7 / 15$.\\

\textbf{Декодирование кодов БЧХ.}\\
Рассматриваем $[n,k,d]$-код БЧХ, длины $n=2^t - 1$, поле
$F = \mathbb{F}_2^t = \mathbb{F}_2[x]/(a(x)),~~\deg a(x) = t$, $\alpha$ - примитивный.

Пусть при передаче кодового слова произошло
$\nu \le r = \lfloor (d - 1)/2 \rfloor$ ошибок в позициях
$j_1, \ldots, j_\nu$.

\underline{Полиномом ошибок} в данном случае будет
$e(x) = x^{j_1} + x^{j_2} + \dots + x^{j_\nu}$.

Вычислим синдромы $s_i = w(\alpha^i) = e(\alpha^i),~ i=\overline{1,~2r}$ и
запишем их через степени $\alpha$. Предполагаем, что ошибки произошли.
\begin{equation*}\begin{cases}
  s_1 = \alpha^{j_1} + \alpha^{j_2} + \ldots + \alpha^{j_\nu},\\
  s_2 = (\alpha^{j_1})^{2} + (\alpha^{j_2})^{2} + \ldots + (\alpha^{j_\nu})^{2},\\
  \dots\\
  s_{2r} = (\alpha^{j_1})^{2r} + (\alpha^{j_2})^{2r} + \ldots + (\alpha^{j_\nu})^{2r},\\
\end{cases}\end{equation*}

Имеем $\nu + 1$ неизвестных в данной системе - $\nu,~j_1,\ldots,j_\nu$.

\textbf{Локатор ошибок}, назовем так
$\beta_i = \alpha^{j_i},~i = \overline{1,\nu}$.
\begin{equation*}\begin{cases}
  s_1 = \beta_{1} + \beta_{2} + \ldots + \beta_{\nu},\\
  s_2 = (\beta_{1})^{2} + (\beta_{2})^{2} + \ldots + (\beta_{\nu})^{2},\\
  \dots\\
  s_{2r} = (\beta_{1})^{2r} + (\beta_{2})^{2r} + \ldots + (\beta_{\nu})^{2r},\\
\end{cases}\end{equation*}

Такая система задает \textit{симметрический полином}.

\textbf{Полиномом локаторов ошибок} назовем
$ \sigma = \prod\limits_{i=1}^{\nu} (1 + \beta_i x) =
1 + \sigma_1 x + \sigma_2 x^2 + \dots + \sigma_\nu x^\nu $.

Формально считаем, $\sigma_0 = 1,~\sigma_i = 0,~i > \nu$. Понятно, что в поле
$\mathbb{F}_2$ корнями такого многочлена будут
$\beta^{-1} = \alpha^{-j_i},~i = \overline{1, \nu}$.\\

Теоремой Виета можем связать $\sigma_i$ и $\beta_i$:
\begin{equation*}\begin{cases}
  \sigma_1 = \beta_{1} + \beta_{2} + \ldots + \beta_{\nu},\\
  \sigma_2 = \beta_1\beta_2 + \beta_2\beta_3 + \beta_1\beta_3
    + \ldots + \beta_{\nu-1}\beta_\nu,\\
  \dots\\
  \sigma_\nu = \beta_1\beta_2\ldots\beta_\nu
\end{cases}\end{equation*}

Последние две системы задают величины синдромов и коэффициентов полинома
локаторов ошибок как значения \textit{симметрических полиномов}: первая - степенных
сумм, вторая - элементарных. Для такого соотношения есть тождества
Ньютона-Жирара, последние $2r - \nu$ из которых в нашем случае записываются как:
\begin{equation*}\begin{cases}
  s_{\nu+1} + \sigma_1s_\nu + \dots + \sigma_{\nu-1}s_2 + \sigma_\nu s_1 = 0,\\
  s_{\nu+2} + \sigma_1s_{\nu + 1} + \dots + \sigma_{\nu-1}s_3 + \sigma_\nu s_2 = 0,\\
  \dots\\
  s_{2r} + \sigma_1s_{2r+1} + \dots + \sigma_{\nu-1}s_{2r-\nu+1} + \sigma_\nu s_{2r - \nu} = 0,\\
\end{cases}\end{equation*}

Эти тождества представляют собой СЛАУ относительно $\sigma_1,\ldots,\sigma_\nu$,
которую можно представить в виде матрицы. Стандартными методами такая система не
решается, так как неизвестно значение $\nu$.\\

Алгоритмы решения системы выше называются \textbf{декодерами}. Самым банальным
декодером является декодер Питерсона прямого решения, который заключается в
последовательном переборе всех $\nu = \overline{r,1}$ пока матрица системы не
окажется невырожденной.\\

Результатом работы декодера является полином локаторов ошибок
$\sigma(x),~\nu = \deg \sigma(x)$. Но это ещё не конец, нужно отыскать все $\nu$
его корней. Для этого можно перебрать все $\alpha,\alpha^2,\ldots,\alpha^n$
мультипликативной группы $\mathbb{F}^*$ и по ним найти позиции ошибок: если $\alpha^i$
корень, то позиция ошибки $j$ есть $j = -i \bmod n$.\\

\textbf{Алгоритм декодирования $[n,k,d]$-кода БЧХ.}\\
\begin{enumerate}
  \item Найти все синдромы $s_i = w(\alpha^i),~i=\overline{1,~d-1}$. Если они
    равны нулю, то см. последний пункт.
  \item Используя декодер найти полином ошибок $\sigma(x)$, его степень -
    количество произошедших ошибок $\nu$.

  \item Найти все корни $\sigma(x)$, например перебором всех элементов $F^*$.

  \item Найти позиции ошибок по степеням корней.
  \item Найти полином ошибок $e(x)$ по найденным позициям и восстановить кодовое
    слово $v(x) = w(x) + e(x)$.
  \item по $v(x)$ восстановить сообщение $u(x)$.
\end{enumerate}

Более менее адекватным декодером является \textbf{декодер Сугиямы} основанный на
обобщенном алгоритме Евклида.


% -------- source --------
\bigbreak
[\cite[page 112-131]{gurov_23}]
