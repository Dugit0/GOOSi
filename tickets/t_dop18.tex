\subsection*{DOP 18 Теоретические основы передачи данных, физический уровень стека протоколов. Системы передачи данных Ethernet и Wi-Fi: алгоритмы работы, управление множественным доступом к каналу}

\textbf{Теоретические основы}


\textbf{Данные} --- описание фактов, явлений.

\textbf{Сигнал} --- представление данных при передаче.

\textbf{Передача} --- процесс взаимодействия передатчика и приёмника, с целью передачи сигнала.
Данные и сигналы могут быть аналоговым (непрерывными) и цифровыми.
Для цифровой передачи данных сигнал нужно разбивать на уровни (для кодирования).

\textbf{Полоса пропускания канала} --- спектр частот, которые канал пропускает без существенного понижения мощности сигнала.

Скорость передачи зависит от способа кодирования данных на физическом уровне и \textbf{сигнальной скорости} -- скорости изменения значения сигнала.

\textbf{Пропускная способность канала} --- максимальная скорость, с которой канал способен передавать данные.

\bigbreak
\textbf{Теорема Найквиста-Котельникова}: $R_{max\_data\_rate} = 2*D*log_2{L}$ bps (bits per second)

- $R_{max\_data\_rate}$ - максимальная пропускная спсобность канала

- $D$ - ширина полосы пропускания канала (максимальная частота сигнала в спектре).

- $L$ - количество уровней (значений) сигнала.

\bigbreak
\textbf{Теорема Котельникова} --- Аналоговый сигнал $u(t)$, не содержащий частот выше $F_{max}$(Гц), полностью определяется последовательностью своих значений в моменты времени, отстоящие друг от друга на $\frac{1}{2F_{max}}$

\bigbreak
\textbf{Шум в канале} измеряется, как соотношение мощности полезного сигнала к мощности шума: $\frac{S}{N}$ (измеряется в децибелах $1dB = 10*\log_{10}{\frac{S}{N}}$)

\bigbreak
\textbf{Теорема Шеннона}: $R_{max} = D*log_2{(1+\frac{S}{N})}$ bps (bits per second)

- $R_{max}$ - максимальная скорость передачи данных по каналу с шумом

- $D$ - ширина полосы пропускания канала (максимальная частота сигнала в спектре).

- $\frac{S}{N}$ - соотношение сигнал-шум в канале,  $S$ - мощность сигнала, $N$ - мощность шума

\textbf{Характеристики физической среды}:
\begin{itemize}
    \item Полоса пропускания
    \item Пропускная способность
    \item Задержка
    \item Затухание
    \item Помехоустойчивость
    \item Достоверность передачи
    \item Стоимость
    \item Простота прокладки
    \item Сложность обслуживания
\end{itemize}


\textbf{\textbf{Ethernet}}

\begin{itemize}
    \item Узлы в сети Ethernet адресуются при помощи 6-байтового двоичного числа, называемого МАС-адресом (Media Access Control — управление доступом к носителю).
    \item Распределением МАС-адресов между производителями оборудования занимается международная некоммерческая организация IEEE (Institute of Elecrical and Electronics Engineers — Институт инженеров электротехники и электроники).
    \item Любой участник может послать в сеть сообщение, но только тогда, когда в ней <тихо> --- нет другой передачи.
    \item Ethernet использует протокол разрешения адресов (ARP) для определения MAC-адресов и их сопоставления с известными адресами сетевого уровня.
    \item У каждого узла в IP-сети есть МАС и IP-адреса.
    \item Узел должен использовать собственные МАС- и IP-адреса в полях источника, а также предоставить МАС и IP-адреса для назначения.
   \item  Несмотря на то, что IP-адрес назначения будет предоставлен более высоким уровнем OSI, отправляющему узлу необходимо найти MAC-адрес назначения для данного канала Ethernet.
    В этом заключается назначение протокола ARP.
\end{itemize}

\textbf{\textbf{Wi-Fi}}

\begin{itemize}
    \item Обычно схема сети Wi-Fi содержит не менее одной точки доступа и не менее одного клиента.
    \item Также возможно подключение двух клиентов в режиме точка-точка (Ad-hoc), когда точка доступа не используется, а клиенты соединяются посредством сетевых адаптеров <напрямую>.
    \item Точка доступа передаёт свой идентификатор сети (SSID -- Service Set Identifier) с помощью специальных сигнальных пакетов
    \item Зная SSID сети, клиент может выяснить, возможно ли подключение к данной точке доступа.
    \item При попадании в зону действия двух точек доступа с идентичными SSID приёмник может выбирать между ними на основании данных об уровне сигнала.
    \item Стандарт Wi-Fi даёт клиенту полную свободу при выборе критериев для соединения.
\end{itemize}

\textbf{\textbf{Управление множественным доступом}}

\begin{itemize}
    \item Проблема управления множественным доступом встаёт в тот момент, когда несколько отправителей хотят отправить свои данные через сеть.
    \item Протокол множественного доступа может определять и предотвращать коллизии пакетов (кадров) данных при условии, что в качестве режима конкурирующего доступа используется метод доступа к каналу, или зарезервированы ресурсы для установления логического канала.
    \item Механизм множественного доступа основан на схеме мультиплексирования (передача нескольких потоков данных с меньшей скоростью по одному каналу связи) физического уровня.
\end{itemize}
\textbf{Протоколы}:
\begin{itemize}
    \item \textbf{CSMA/CD (Carrier Sense Multiple Access with Collision Detection)}
    Если во время передачи кадра рабочая станция обнаруживает другой сигнал, занимающий передающую среду, она останавливает передачу, посылает сигнал преднамеренной помехи и ждёт в течение случайного промежутка времени (backoff delay), перед тем как снова отправить кадр.
    В Ethernet коллизии могут быть обнаружены сравнением передаваемой и получаемой информации. Если она различается, то другая передача накладывается на текущую (возникла коллизия) и передача прерывается немедленно. Посылается сигнал преднамеренной помехи, что вызывает задержку передачи всех передатчиков на произвольный интервал времени, снижая вероятность коллизии во время повторной попытки.
    Ethernet является классическим примером протокола CSMA/CD.
    \item \textbf{Aloha (чистая)}
    Отправитель начинает передавать, если в этот момент он слышит чужой сигнал, то он понимает, что произошло наложение и перестает передавать, ждет случайное время и начинает передавать сначала.
    В Wi-Fi суть та же.
    \item \textbf{Aloha (слотированная)}
    Модификация алохи. Все время работы канала разделяется на слоты. Размер слота при этом должен быть равен максимальному времени кадра. Такая организация работы канала требует синхронизации: одна из станций испускает сигнал начала очередного слота, поскольку передачу теперь можно начинать не в любой момент, а только по специальному сигналу, то время на обнаружение коллизии сокращается вдвое. Размер слота при этом должен быть равен максимальному времени кадра.
\end{itemize}

% -------- source --------
\bigbreak
[\cite[page 69-96]{smelbook2}]
