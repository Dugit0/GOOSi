\subsection*{DOP 1 Теорема Поста о полноте систем функций в алгебре логики.}


\textbf{Полная система} (в $P_2$) --- множество $\mathcal{A}$ ФАЛ такое, что любую ФАЛ можно выразить формулой над $\mathcal{A}$.

$\mathLet ~ \mathcal{A} \subseteq P_2$. Тогда \textbf{замыкание} $\mathcal{A}$ (обозн. $[\mathcal{A}]$) --- множество всех ФАЛ, которые можно выразить формулами над $\mathcal{A}$.

Класс $\mathcal{A}$ называется \textbf{замкнутым}, если $\mathcal{A} = [\mathcal{A}]$.

Говорят, что набор $\tilde{\alpha} = (\alpha_1, ... \alpha_n)$ \textbf{предшествует} набору $\tilde{\beta} = (\beta_1, ... \beta_n)$,  $\tilde{\alpha} \preccurlyeq \tilde{\beta}$ ($\tilde{\alpha}$ не больше $\tilde{\beta}$), если $\alpha_i \leqslant \beta_i$ для всех $i = 1, ..., n$.

Если $\tilde{\alpha} \preccurlyeq \tilde{\beta}$ и $\tilde{\alpha} \neq \tilde{\beta}$, то говорят, что набор $\tilde{\alpha}$ \textbf{строго предшествует} набору $\tilde{\beta}$, $\tilde{\alpha} \prec \tilde{\beta}$.

Наборы $\tilde{\alpha}$ и $\tilde{\beta}$ называются сравнимыми, если $\tilde{\alpha} \preccurlyeq \tilde{\beta}$ либо $\tilde{\beta} \preccurlyeq \tilde{\alpha}$.

Например, вектора $[0, 1]$ и $[1, 0]$ -- несравнимы.

Стандартные замкнутые классы:
\begin{itemize}
    \item[--] $T_0 = \{f \in P_2 | f(0, \dots, 0) = 0\}$; -- сохраняющие 0
    \item[--] $T_1 = \{f \in P_2 | f(1, \dots, 1) = 1\}$; -- сохраняющие 1
    \item[--] $L = \{f(x_1, \dots, x_n) = a_0 \oplus a_1 x_1 \oplus \dots \oplus a_n x_n\}$ --- линейные функции;
    \item[--] $S = \{f(x_1, \dots, x_n) = f^*(x_1, \dots, x_n) = \bar{f}(\overline{x_1}, \dots, \overline{x_n})\}$ --- самодвойственные функции;
    \item[--] $M = \{\alpha \leqslant \beta \rightarrow f(\alpha) \leqslant f(\beta)\}$ --- монотонные функции;
\end{itemize}

\textit{Вспомогательные леммы:}
\begin{enumerate}
    \item Если булева функция $f$ немонотонна, то из нее подстановкой вместо аргументов констант 0 и 1 и переменной $x$ можно получить $\bar{x}$.
    
    \begin{proof}
        Немнонотонна $\implies \exists$ два набора $\alpha < \beta$, а $f(\alpha) = 1 > f(\beta) = 0$. Будем заменять в $\alpha$ 0 на 1 по одному, чтобы получился $\beta$, промежуточные назовем $\alpha_k$, $\alpha_0=\alpha, \alpha_r=\beta$. В какой-то момент получим $f(\alpha_k) = 1, f(\alpha_{k+1} = 0)$, получили два соседних набора $\alpha_k$ и $\alpha_{k+1}$. Пусть различаются в i-й переменной. Заменим в $\alpha_k$ i-ую переменную на x, получим $\tilde{\alpha}_k$. $f(\tilde{\alpha}_k) = \overline{x}$
    \end{proof}
    \item Если булева функция несамодвойственна, то из неё подстановкой вместо аргументов переменной $x$ и её отрицания $\bar{x}$ можно получить либо константу 0, либо константу 1.
    
    \begin{proof}
        Несамодвойственна, то $\exists \alpha$ т.ч. $f(\alpha)=f(\overline{\alpha})=C$. Рассмотрим $\phi(x) = f(x \oplus \alpha_1, \dots, x \oplus \alpha_n)$. Тогда в зависимости от x в аргументах функции набор $\alpha$ или $\overline{\alpha}$, в любом случае $\phi(x) = C$.
    \end{proof}
    \item Если булева функция нелинейная, то из нее подстановкой вместо аргументов констант, переменных $x$, $y$, их отрицаний $\bar{x}$, $\bar{y}$ можно получить $x \cdot y$ или $\overline{x \cdot y}$.
    
    \begin{proof}
        Рассмотрим полином Жегалкина $P_f$ функции $f$(представление в виде $\oplus$ из конъюнкций). В нем найдется слагаемое, которое конъюникция двух или более переменных, пусть $x_1 \cdot x_2 \cdot \dots \cdot x_r$.
        
        $P_f = x_1 \cdot x_2 \cdot g_1(x_3, \dots, x_n) \oplus x_1 \cdot g_2(x_3, \dots, x_n) \oplus x_2 \cdot g_3(x_3, \dots, x_n) \oplus g_4(x_3, \dots, x_n)$
        
        Либо $g_1 = 1$(если r = 2), либо $\exists \alpha$ т.ч. $g_1(\alpha) = 1$. Обозначим $g_2(\alpha) = a, g_3(\alpha) = b, g_4(\alpha) = c$.
        
        $\phi(x, y) = f(x \oplus b, y \oplus a, \alpha_3, \dots , \alpha_n) = (x \oplus b)(y \oplus a) \oplus a(x \oplus b) \oplus b(y \oplus a) \oplus c = $\{раскройте сами\} $xy \oplus d$, d-константа.
    \end{proof}
\end{enumerate}

\textbf{Теорма Поста.}
Система ФАЛ $\mathcal{A} = \{f_1, f_2, \dots\}$ является полной в $P_2$ $\iff$ она не содержится целиком ни в одном из следующих классов: $T_0,~T_1,~S,~L,~M$.

\begin{proof}
\textit{Необходимость.} 
Пусть $\mathcal{A}$ --- полная система, $N$ --- любой из классов $T_0,~T_1,~S,~L,~M$ , и (от противного) пусть $\mathcal{A} \subseteq N$.
Тогда $[\mathcal{A}] \subseteq [N] = N \neq P_2$, то есть $[\mathcal{A}] \neq P_2$.
Полученное противоречие завершает обоснование необходимости.

\textit{Достаточность.}
Пусть $\mathcal{A}$ не является подмножеством ни одного из этих классов. 
Тогда в $\mathcal{A}$ существуют функции $f_0 \notin T_0,~f_1 \notin T_1,~f_L \notin L,~f_M \notin M,~f_S \notin S$. Пусть $\mathcal{B} = \{f_0,~f_1,~f_M,~f_L,~f_S\}$ Достаточно показать, что $[\mathcal{A}] \supseteq [\mathcal{B}] = P_2$.

Выразим формулами над $\mathcal{B}$ все функции из полной системы $\{0,1,\overline{x}, x \cdot y\}$.
\begin{itemize}
    \item \textit{Получение отрицания.} 
    Рассмотрим функцию $f_0(x_1, \dots ,x_n) \notin T_0$ и введём функцию $\varphi_0(x) = f_0(x, x, \dots, x)$. 
    Так как функция $f_0$ не сохраняет нуль, $\varphi_0(0) = f_0(0, 0, \dots, 0) = 1$. 
    Возможны два случая: либо $\varphi_0(x) = \bar{x}$, либо $\varphi_0(x) \equiv 1$.
    
    Рассмотрим функцию $f_1(x_1, \dots, x_n) \notin T_1$ и введём функцию $\varphi_1(x) = f_1(x, x, \dots, x)$. 
    Так как функция $f_1$ не сохраняет единицу, $\varphi_1(1) = f_1 (1, 1, \dots, 1) = 0$. 
    Возможны два случая: либо $\varphi_1 = \bar{x}$, либо $\varphi_1(x) \equiv 0$.
    
    Если хотя бы в одном случае получилось искомое отрицание, пункт завершен.
    Если же в обоих случаях получились константы, то согласно лемме о немонотонной функции, подставляя в функцию вместо всех переменных константы и тождественные функции, можно получить отрицание.
    
    Отрицание получено.
    \item \textit{Получение констант 0 и 1.}
    Имеем $f_S \notin S$. 
    Согласно лемме о несамодвойственной функции, подставляя вместо всех переменных функции $f_S$ отрицание (которое получено в пункте 1) и тождественную функцию, можно получить константы: $[f_S , \bar{x}] \supseteq [0, 1]$. 
    
    Константы получены.
    \item \textit{Получение конъюнкции.} 
    Имеем функцию $f_L \notin L$. 
    Согласно лемме о нелинейной функции, подставляя в функцию вместо всех переменных константы и отрицания (которые были получены на предыдущих шагах доказательства), можно получить либо конъюнкцию, либо отрицание конъюнкции.
    Однако на первом этапе отрицание уже получено, следовательно, всегда можно получить конъюнкцию: $[f_L, 0, 1, \bar{x}] \supseteq \{x \cdot y, \overline{x \cdot y}\}$. 
    
    Конъюнкция получена.
\end{itemize}

В результате получено, что $[f_0, f_1, f_L, f_S, f_M] \supseteq [0, 1, \bar{x}, x \cdot y] = P_2$.
Что и требовалось доказать.
\end{proof}


% -------- source --------
\bigbreak
[\cite{mkpres}]
