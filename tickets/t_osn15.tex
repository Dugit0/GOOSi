\subsection*{OSN 15 Характеристический многочлен линейного оператора. Собственные числа и собственные векторы.}

\textbf{Линейным оператором} векторного пространства $V$ над полем $P$ в векторное пространство $W$ над тем же полем $P$ называется оператор $\mathcal{A} \in \mathcal{L}(V, W)$, удовлетворяющий условию линейности для всех $x, y \in V$ и $\alpha \in P$:
$$\mathcal{A} (x+y) = \mathcal{A} x + \mathcal{A} y$$
$$\mathcal{A} (\alpha x) = \alpha \mathcal{A} x$$

Пусть $e = (e_1, ..., e_n)$ и $f = (f_1, ..., f_m)$ -- базисы пространств $V$ и $W$. Линейный оператор $\mathcal{A} \in \mathcal{L}(V, W)$ однозначно определяется заданием векторов $\mathcal{A} e_1, ..., \mathcal{A} e_n$. В свою очередь вектора $\mathcal{A} e_i, i \in 1..n$ своими координатами в базисе $f$, т.е. коэффициентами разложений

$$\begin{cases} 
 \mathcal{A} e_1 = a_{11} f_1 + a_{21} f_2 + ... + a_{m1} f_m \\
 ... \\
 \mathcal{A} e_n = a_{1n} f_n + a_{2n} f_2 + ... + a_{mn} f_m
\end{cases}$$

Матрица $A_{fe} = 
\begin{bmatrix}
a_{11} & a_{12} & ... & a_{1n} \\
... & ... & ... & ... \\
a_{m1} & a_{m2} & ... & a_{mn} \\
\end{bmatrix}$
называется \textbf{матрицей линейного оператора} $\mathcal{A}$ в базисе $e$ и $f$. Матрицы линейного оператора в различных парах базисов эквивалентны.

Матрицы $A, B \in P^{m*n}$ называются \textbf{эквивалентными}, если существуют м-цы $D \in P^{m*m}, |D| \neq 0$ и $Q \in P^{n*n}, |Q| \neq 0$, такие что $B = D A Q$.

Квадратные матрицы $A, B \in P^{n*n}$ называются \textbf{подобными}, если существует м-ца $D \in P^{n*n}, |D| \neq 0$, такая что $B = D^{-1} A D$.

% ___думаю, это тут не нужно, в Киме есть теорема, но доказывать ее не хочется
% Подобные матрицы при задании одного и того же линейного оператора матрицей в разных координатных системах. При этом м-ца $D$ называется матрицей перехода от одной системы к другой.

\centerline{\textbf{Собственные числа и собственные векторы.}}

Пусть $V$ -- линейное пространство над полем $P$ и $\mathcal{A} \in \mathcal{L}(V, V)$.

Вектор $x: x\neq \theta, x \in V$ называется \textbf{собственным вектором} оператора $\mathcal{A}  = \mathcal{L}(V, V)$, если $\exists \lambda \in P$, такой что : 
$$\mathcal{A}x = \lambda x$$
$\lambda$ называется \textbf{собственным значением} оператора $\mathcal{A}$, соответствующим собственному вектору $x$.

\textbf{Спектр оператора} $\mathcal{A}$ -- это множество всех собственных значений $\mathcal{A}$.

\textbf{Теорема}: Собственные векторы, отвечающие различным собственным значениям, линейно независимы.

\begin{proof}
По индукции, для $n=1$ верно, т.к. с.в. $\neq \theta$ по определению. Пусть верно для из $n-1$ векторов, докажем для $n$.
От противного: пусть $x_1,\dots,x_n$ --- собственные векторы, отвечающие различным собственным значениям $\lambda_1,\dots,\lambda_n$, --- линейно зависимы. 
Тогда существует нетривиальная линейная комбинация этих векторов:
$$ \alpha_1x_1+\dots+\alpha_nx_n=\theta$$
Подействуем на неё оператором $A:~\alpha_1\lambda_1x_1+\dots+\alpha_n\lambda_nx_n=\theta$

Домножим линейную комбинацию на $\lambda_n:~\alpha_1\lambda_nx_1+\dots+\alpha_n\lambda_nx_n=\theta$

Вычтем первое равенство из второго, получим:
$$\alpha_1(\lambda_n-\lambda_1)x_1+\dots+\alpha_{n-1}(\lambda_n-\lambda_{n-1})x_{n-1}=\theta$$
Получили нетривиальную линейную комбинацию линейно независимых векторов, значит предположение неверно и теорема верна.
\end{proof}

\textbf{Следствие:} Линейный оператор, действующий в $n$-мерном пространстве, не может иметь более $n$ различных собственных значений.


\centerline{\textbf{Характеристический многочлен.}}

$f(\lambda) = |A - \lambda I|$ --- \textbf{характеристический многочлен} матрицы $A$.

Уравнение $det(\mathcal{A} - \lambda I) = 0$ --- \textbf{характеристическое уравнение} оператора $\mathcal{A}$.

\textbf{Теорема}: Характеристические многочлены подобных матриц совпадают, т.е.
    $A = Q^{-1}BQ \implies |A - \lambda I| = |B - \lambda I|$.

\begin{proof}
$|A - \lambda I| = |Q^{-1} B Q - \lambda I| = |Q^{-1} B Q - Q^{-1} \lambda I Q| = |Q^{-1} (B - \lambda I) Q| = |B - \lambda I|$. 

Т.е. $\forall \lambda$ характеристические многочлены м-ц $A$ и $B$ принимают одинаковые значения.
\end{proof}

\textbf{Следствие:} Все матрицы одного и того же линейного оператора имеют одинаковые характеристические многочлены.

\bigbreak

\textbf{Образом} лин. оператора $\mathcal{A} \in \mathcal{L}(V, W)$ наз-ся мн-во $im\mathcal{A} = \{y \in W | \exists x \in V : \mathcal{A}x=y \}$.

\textbf{Ядром} лин. оператора $\mathcal{A} \in \mathcal{L}(V, W)$ наз-ся мн-во $ker\mathcal{A} = \{x \in V | \mathcal{A}x=0 \}$.

\textbf{Теорема}: $\lambda$ --- собственное значение оператора $\mathcal{A} \iff \lambda$ --- корень характеристического уравнения оператора $\mathcal{A}$.

\begin{proof}
 $(\implies):~\lambda \in P$ --- собственное значение, $\exists x:~Ax = \lambda x,~x\neq 0,~(A - \lambda I)x = 0 \implies
    dim(ker(A - \lambda I))\neq 0 \implies det(A - \lambda I) = 0$.

    $(\impliedby):~\exists\lambda$ --- корень характеристического уравнения $\implies dim(im(A - \lambda I))
    \leqslant n - 1 \implies dim(ker(A - \lambda I)) \geqslant 1 \implies \exists x\neq\theta : (A - \lambda I)x = 0$.
\end{proof}



% -------- source --------
\bigbreak
[\cite[page 240-260]{kim}]
