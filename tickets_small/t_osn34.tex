\subsection*{OSN 34 Постановка краевых задач для уравнения теплопроводности.  Метод разделения переменных для решения первой краевой задачи.}

Краевые задачи для уравнения теплопроводности представляют собой математические модели процессов распространения тепла, например, в стержне.
    
$u(x,t)$ --- температура в сегменте с координатами $x$ во время $t$.

$F(x, t)$ --- плотность тепловых источников, $a^2 = \frac{k}{c\rho}$ --- коэффициент температуропроводности, $f(x,t)=\frac{F(x,t)}{c\rho}$ , $c$ --- удельная теплоемкость, $k$ --- коэффициент теплопроводности, $\rho$ --- плотность.

\bigbreak

Одномерное уравнение теплопроводности:

$u_t(x,t)=a^2u_{xx}(x,t)+f(x,t)$


\textbf{Основные типы задач:}
\begin{itemize}
    \item \textbf{Первая краевая задача.}
    
    $\begin{cases}
    u_t(x,t)=a^2u_{xx}(x,t)+f(x,t),~0<x<l,~t>0&\\
    u(0,t)=\mu_1(t),~t\geqslant0&\\
    u(l,t)=\mu_2(t)&\\
    u(x,0)=\varphi(x),~0\leqslant x\leqslant l&\\
    \end{cases}$
    \item \textbf{Вторая краевая задача.}
    
    $\begin{cases}
    u_t(x,t)=a^2u_{xx}(x,t)+f(x,t),~0<x<l,~t>0&\\
    u_x(0,t)=\nu_1(t),~t\geqslant0&\\
    u_x(l,t)=\nu_2(t)&\\
    u(x,0)=\varphi(x),~0\leqslant x\leqslant l&\\
    \end{cases}$
    \item \textbf{Смешанная краевая задача.} --- одно из краевых условий задано функцией $u(0,t)$ или $u(l,t)$, а другое производной $u$ по $x$.
\end{itemize}

$u(x,t)$ --- \textbf{решение 1-ой краевой задачи} для уравнения теплопроводности, если:
\begin{enumerate}
    \item $u(x,t)\in C([0,l]\times[0,+\infty))$;
    \item $u(x,t)\in C^2((0,l)\times(0,+\infty))$;
    \item $u(x,t)$ удовлетворяет условиям 1-ой краевой задачи.
\end{enumerate}

\bigbreak
Далее будем рассматривать $f(x,t)=0$.

\bigbreak
\textbf{Метод разделения переменных для решения первой краевой задачи.}

$\begin{cases}
    u_t(x,t)=a^2u_{xx}(x,t),~0<x<l,~t>0&\\
    u(0,t)=0,~t\geqslant0&\\
    u(l,t)=0&\\
    u(x,0)=\varphi(x),~0\leqslant x\leqslant l&\\
\end{cases}$

\bigbreak
Будем искать решение в виде $u(x,t)=X(x)T(t)$. Рассмотрим задачу с однородными начальными и краевыми условиями:

$\begin{cases}
XT'=a^2X''T&\\
X(0)T(t)=0&\\
X(l)T(t)=0&\\
\end{cases} \rightarrow 
\begin{cases}
\frac{T'}{a^2T}=\frac{X''}{X}=-\lambda&\\
X(0)=X(l)=0&\\
\end{cases}$

\bigbreak
Получаем две задачи:
\begin{enumerate}
    \item Задача Штурма-Лиувилля:
    
    $\begin{cases}
    X''+\lambda X=0&\\
    X(0)=X(l)=0&\\
    \end{cases}$
    
    Рассматриваем 3 случая: $\lambda<0,~\lambda=0,~\lambda>0$. При $\lambda>0$ получаем $\lambda_n=(\frac{\pi n}{l})^2,~X_n(x)=\sin(\frac{\pi n x}{l}),~n\in\mathbb{N}$
    
    \item $T'+(\frac{\pi n a}{l})^2 T = 0$
    
    Решение: $T_n(t)=a_n e^{-(\frac{\pi n a}{l})^2t},~n\in\mathbb{N}$
\end{enumerate}

\bigbreak
Получаем решение: $u(x,t)=\displaystyle\sum_{n=1}^{\infty} a_n e^{-(\frac{\pi n a}{l})^2t} \sin(\frac{\pi n x}{l})$

Для нахождения $a_n$ используем начальное условие.
Так как $\{\sin(\frac{\pi n x}{l})\}$ --- замкнутая полная система функций, то $\forall$ кусочно-дифференцируемую функцию можно разложить в ряд Фурье:

$$\varphi(x) = \displaystyle\sum_{n=1}^{\infty} \varphi_n \sin(\frac{\pi n x}{l}),~\varphi_n=\frac{2}{l}\int\limits_0^l\varphi(\xi)\sin(\frac{\pi n \xi}{l})d\xi$$

Так как $u(x,0) = \varphi(x)$, получаем $a_n = \varphi_n,~n\in\mathbb{N}$. Таким образом, получаем решение:

$$u(x,t)=
\displaystyle\sum_{n=1}^{\infty} \frac{2}{l}\int\limits_0^l\varphi(\xi)\sin \left( \frac{\pi n \xi}{l} \right)d\xi \cdot e^{-(\frac{\pi n a}{l})^2t} \sin \left(\frac{\pi n x}{l} \right)$$

\bigbreak
Для существования решения достаточно потребовать, чтобы $\varphi\in C^2[0,l]$ и $\varphi(0)=\varphi(l)=0$.

\bigbreak
Остальные случаи:
\begin{itemize}
    \item $X'(0)=X(l)=0$: $\lambda_n=\left(\frac{\pi(2n+1)}{2l}\right)^2, X_n=\cos\sqrt{\lambda_n}, n=0,1,2,\dots$ \\
    \item $X(0)=X'(l)=0$: $\lambda_n=\left(\frac{\pi(2n+1)}{2l}\right)^2, X_n=\sin\sqrt{\lambda_n}, n=0,1,2,\dots$ \\
    \item $X'(0)=X'(l)=0$: $\lambda_n=\left(\frac{\pi n}{l}\right)^2, X_n=\cos\sqrt{\lambda_n}, n=1,2,\dots; X_0=1$ \\
\end{itemize}

% -------- source --------
\bigbreak
[\cite[page 200-202]{urmati_tikhonov}]
