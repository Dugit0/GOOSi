\subsection*{OSN 33 Задача Коши для уравнения колебания струны. Формула Даламбера.}

\textbf{Задача Коши для уравнения колебания струны.}

$$\begin{cases}
u_{tt}=a^2u_{xx}+f(x,t)&\\
u(x,0)=\varphi(x)&\\
u_t(x,0)=\psi(x)&\\
\end{cases}$$ где $t>0,~a>0,~u(x,t)\in C^2(t > 0, x\in \mathbb{R})\cap C^1(t \geqslant 0, x\in \mathbb{R})$.

\bigbreak
\textit{Физическая интерпретация:} уравнение малых поперечных колебаний струны. $u(x,t)$ -- положение точки струны с координатой $x$ в момент времени $t$. Если концы струны закреплены, то $u(0,t)=0, u(l,t)=0$. Так как процесс колебаний струны зависит от ее начальной формы и распределения скоростей, то задают начальные условия. $a = \sqrt{\frac{T_0}{\rho}}$, где $T_0$ -- величина натяжения (не зависит от $x$, $\rho$ -- линейная плотность струны. $f(x,t)$ -- плотность внешних сил.

\bigbreak
Далее будем рассматривать $f(x,t) = 0$.
\bigbreak

(\textit{Представим что $\frac{\partial^2u}{\partial t^2} = \frac{d^2u}{dt^2}$ и $\frac{\partial^2u}{\partial x^2} = \frac{d^2u}{dx^2}$:
$\frac{d^2u}{dt^2} = a^2\frac{d^2u}{dx^2} \implies$
${d^2udx^2 = a^2dt^2d^2u} \implies$
${dx^2 = a^2dt^2}$.})
Характеристическое уравнение: $dx^2-a^2dt^2=0$. \\
$dx - adt=0, ~ dx+adt = 0 \implies x-at=C_1 = const, ~ x+at=C_2 = const$

Сделаем замену переменных: $$x+at=\xi,~x-at=\eta$$
$$\xi_x=1, ~ \xi_{xx}=0, ~ \eta_x = 1,  ~ \eta_{xx} = 0,$$
$$\xi_t=a, ~ \xi_{tt}=0, ~ \eta_t = -a, ~ \eta_{tt} = 0.$$

Тогда:
$$u_x=u_{\xi}\cdot\xi_x+u_{\eta}\cdot\eta_x,$$
$$u_t=u_{\xi}\cdot\xi_t+u_{\eta}\cdot\eta_t,$$
$$u_{xx}=u_{\xi\xi}\cdot\xi_x^2+2u_{\xi\eta}\cdot\xi_x\cdot\eta_x+u_{\xi}\cdot\xi_{xx}+u_{\eta\eta}\cdot\eta_x^2+u_{\eta}\cdot\eta_{xx},$$
$$u_{tt}=u_{\xi\xi}\cdot\xi_t^2+2u_{\xi\eta}\cdot\xi_t\cdot\eta_t+u_{\xi}\cdot\xi_{tt}+u_{\eta\eta}\cdot\eta_t^2+u_{\eta}\cdot\eta_{tt}.$$

Подставляем в уравнение $u_{tt}=a^2u_{xx}$:
$$u_{\xi\xi}\cdot\xi_t^2+2u_{\xi\eta}\cdot\xi_t\cdot\eta_t+\displaystyle\underbrace{u_{\xi}\cdot\xi_{tt}}_{\text{= 0}}+u_{\eta\eta}\cdot\eta_t^2+\displaystyle\underbrace{u_{\eta}\cdot\eta_{tt}}_{\text{= 0}}=$$ $$=a^2(u_{\xi\xi}\cdot\xi_x^2+2u_{\xi\eta}\cdot\xi_x\cdot\eta_x+\displaystyle\underbrace{u_{\xi}\cdot\xi_{xx}}_{\text{= 0}}+u_{\eta\eta}\cdot\eta_x^2+\displaystyle\underbrace{u_{\eta}\cdot\eta_{xx}}_{\text{= 0}})$$
Преобразуем:
$$a^2u_{\xi\xi}-2a^2u_{\xi\eta}+a^2u_{\eta\eta}=a^2u_{\xi\xi}+2a^2u_{\xi\eta}+a^2u_{\eta\eta}$$
$$4a^2u_{\xi\eta}=0,   a>0$$
Получаем: $u_{\xi\eta}=0$

Найдем общий интеграл этого уравнения: $u_{\eta}(\xi, \eta)=f^*(\eta)$, где $f^*(\eta)$ - некоторая функция только переменного $\eta$.

Интегрируя это равенство по $\eta$ при фиксированном $\xi$, получим:
\begin{equation}
    u(\xi, \eta)=\int f^*(\eta)d\eta = f_1(\xi) + f_2(\eta)
    \label{isuvdno}
\end{equation}

Обратно, каковы бы ни были дважды дифференцируемые функции $f_1$ и $f_2$, функция $u(\xi,\eta)$, определяемая формулой (\ref{isuvdno}), представляет собой решение уравнения $u_{\xi\eta}=0$. Так как всякое решение уравнения $u_{\xi\eta}=0$ может быть представлено в виде (\ref{isuvdno}) при соответствующем выборе $f_1$ и $f_2$, то формула (\ref{isuvdno}) является общим интегралом этого уравнения. Следовательно, функция $$u(x, t) = f_1(x+at) + f_2(x-at)$$ является общим интегралом уравнения $u_{tt}=a^2u_{xx}$.

Удовлетворим начальным условиям:
$$\begin{cases}
u(x,0)=f_1(x)+f_2(x)=\varphi(x)&\\
u_t(x,0)=-af_1'(x)+af_2'(x)=\psi(x)&\\
\end{cases}
$$

Проинтегрируем второе равенство, получим:
$$\begin{cases}
f_1(x)+f_2(x)=\varphi(x)&\\
f_1(x)-f_2(x)=\frac{1}{a}\int\limits_{x_0}^{x}\psi(\alpha)d\alpha + C&\\
\end{cases}
$$

Сложим и вычтем два полученных равенства:
$$\begin{cases}
f_1(x)=\frac{1}{2}\varphi(x)+\frac{1}{2a}\int\limits_{x_0}^{x}\psi(\alpha)d\alpha + \frac{C}{2}&\\
f_2(x)=\frac{1}{2}\varphi(x)-\frac{1}{2a}\int\limits_{x_0}^{x}\psi(\alpha)d\alpha - \frac{C}{2}&\\
\end{cases}
$$

Подставим в $u(\xi, \eta) = f_1(x+at) + f_2(x-at)$ найденные выражения для $f_1, f_2$:
$$u(x,t)=\frac{\varphi(x+at)+\varphi(x-at)}{2} + \frac{1}{2a}\left(\int\limits_{x_0}^{x+at}\psi(\alpha)d\alpha - \int\limits_{x_0}^{x-at}\psi(\alpha)d\alpha\right)$$

Окончательно:
$$u(x,t)= \frac{\varphi(x+at)+\varphi(x-at)}{2} + \frac{1}{2a}\int\limits_{x-at}^{x+at}\psi(\alpha)d\alpha$$

Полученная формула -- \textbf{формула Даламбера}.

\textbf{Теорема о применимости формулы Даламбера.} Пусть $\varphi(x) \in C^2(-\infty, \infty), \psi \in C^1[0,+\infty), a > 0$. Тогда формула Даламбера представляет единственное решение задачи Коши для уравнения колебания струны.

% -------- source --------
\bigbreak
[\cite[page 50-52]{urmati_tikhonov}]
