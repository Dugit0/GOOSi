\subsection*{OSN 12 Системы линейных алгебраических уравнений. Теорема Кронекера-Капелли. Общее решение системы линейных алгебраических уравнений.}

\textbf{Системой m линейных алгебраических уравнений с n неизвестными} называется совокупность отношений:
    $$\begin{cases}
        a_{11}x_1 + a_{12}x_2+\dots+a_{1n}x_n = b_1&\\
        a_{21}x_1 + a_{22}x_2+\dots+a_{2n}x_n = b_2&\\
        \dots&\\
        a_{m1}x_1 + a_{m2}x_2+\dots+a_{mn}x_n = b_m&\\
    \end{cases}$$
    
Упорядоченная совокупность чисел $c_1, ..., c_n \in\mathbb{R}$ называется \textbf{решением} системы, если при подстановке этих чисел в систему вместо неизвестных $x_1, ..., x_n$ соответственно каждое уравнение обращается в тождество.
Две СЛАУ \textbf{эквиваленты}, если множества их решений совпадают.
СЛАУ \textbf{совместна}, если существует хотя бы одно решение.
СЛАУ называется \textbf{определенной}, если она имеет единственное решение, если имеет больше одного -- \textbf{неопределенной}


\textbf{Теорема.} СЛАУ с квадратной невырожденной матрицей совместна и имеет единственное решение.

\begin{proof}
В силу невырожденности матрицы A для нее существует обратная матрица $A^{-1}$.
Вектор $x = A^{-1}b$ -- решение. Оно единственно. Если $y$ -- другое решение системы, то $Ax \equiv Ay$. Умножив обе части тождества слева на $A^{-1}$, получим $x = y$
\end{proof}

\textbf{Решение СЛАУ с помощью правила Крамера.}

Введём:
$A_i = \begin{pmatrix} a_{1,1} & \dots & a_{1,i-1} & b_1 & a_{1,i+1} & \dots & a_{1,n} \\
           a_{2,1} & \dots & a_{2,i-1} & b_2 & a_{2,i+1} & \dots & a_{2,n} \\
            & & & \dots & & & \\
           a_{n,1} & \dots & a_{n,i-1} & b_n & a_{n,i+1} & \dots & a_{n,n}
\end{pmatrix}$

Тогда: $x_i = \frac{|A_i|}{|A|},~i=1,\dots,n$. $A^{-1}$ получается из матрицы A заменой ее i-го столбца столбцом свободных членов.

$B = (A|b)$ --- \textbf{расширенная} матрица.
\newline
\textbf{Теорема Кронеккера-Капелли.} СЛАУ совместна $\iff$ $rgB = rgA$.

    \begin{proof}
    $(\implies)$ Пусть СЛАУ совместна $\implies \exists x_1, x_2, \dots, x_n:~a_1x_1 + a_2x_2 + \dots + a_nx_n = b \implies$ столбец $b$ является линейной комбинацией столбцов $a_1,\dots,a_n \implies rgB=rg(A|b) = rgA$
    
    $(\impliedby)$ Пусть $rgB = rg(A|b) = rgA = r$. Возьмём в матрице $A$ произвольный базисный минор. Так как $rg(A|b) = r$, то он же будет базисным минором матрицы $(A|b)$. Следовательно, последний столбец матрицы $(A|b)$ будет являться линейной комбинацией столбцов матрицы $A$. Коэффициенты этой комбинации являются решением СЛАУ $Ax=B$, то есть система совместна.
    \end{proof}

Пусть система
$$\begin{cases}
        a_{11}x_1+\dots + a_{1r}x_r+\dots+a_{1n}x_n = b_1&\\
        a_{21}x_1+\dots + a_{2r}x_r+\dots+a_{2n}x_n = b_2&\\
        \dots&\\
        a_{n1}x_1+\dots + a_{nr}x_r+\dots+a_{nn}x_n = b_n&\\
    \end{cases}$$

совместна и $rgA = rgB = r$. Будем считать, что базисный минор матрицы $A$ находится в левом верхнем углу:
$$M = \begin{vmatrix}
          a_{11} & \dots & a_{1r}\\
          & \dots & \\
          a_{r1} & \dots & a_{rr}
    \end{vmatrix} \neq 0 $$
Рассмотрим \textbf{укороченную} систему:
\begin{equation}
    \begin{cases}
        a_{11}x_1+\dots + a_{1r}x_r+\dots+a_{1n}x_n = b_1&\\
        a_{21}x_1+\dots + a_{2r}x_r+\dots+a_{2n}x_n = b_2&\\
        \dots&\\
        a_{r1}x_1+\dots + a_{rr}x_r+\dots+a_{rn}x_n = b_r&\\
    \end{cases}
    \label{eq1}
\end{equation}

\textbf{Теорема.} Укороченная система эквивалентна исходной системе.

\begin{proof}
Обе системы содержат одинаковое число неизвестных. Любое решение \textit{исходной} системы является рещением системы (\ref{eq1}). Покажем, что верно и обратное.
В расширенной матрице исходной системы первые $r$ строк являются базисными. Следовательно, все остальные строки согласно теореме о базисном миноре будут линейными комбинациями этих строк. Это означает, что каждое уравнение исходной системы, начиная с $(r+1)$-го, будет линейной комбинацией первых $r$ уравнений этой системы. Следовательно, каждое решение первых r уравнений исходной системы обращает в тождества все последующие уравнения.
\end{proof}

Запишем систему в виде
$$\begin{cases}
        a_{11}x_1+\dots + a_{1r}x_r = b_1-a_{1,r+1}x_{r+1} -\dots-a_{1n}x_n &\\
        a_{21}x_1+\dots + a_{2r}x_r= b_2-a_{2,r+1}x_{r+1} -\dots-a_{2n}x_n&\\
        \dots&\\
        a_{r1}x_1+\dots + a_{rr}x_r= b_r-a_{r,r+1}x_{r+1} -\dots-a_{rn}x_n&\\
    \end{cases}$$

Придав свободным членам $x_{r+1}, \dots, x_n$ произвольные значения $c_{r+1}, \dots, c_n$, получим систему уравнений относительно неизвестных $x_1, x_2, \dots, x_r$ с квадратной невырожденной матрицей:
\begin{equation}
    \begin{cases}
        a_{11}x_1+\dots + a_{1r}x_r = b_1-a_{1,r+1}c_{r+1} -\dots-a_{1n}c_n &\\
        a_{21}x_1+\dots + a_{2r}x_r= b_2-a_{2,r+1}c_{r+1} -\dots-a_{2n}c_n&\\
        \dots&\\
        a_{r1}x_1+\dots + a_{rr}x_r= b_r-a_{r,r+1}c_{r+1} -\dots-a_{rn}c_n&\\
    \end{cases}
    \label{eq2}
\end{equation}

Эта система имеет единственное решение $c_1, c_2, \dots, c_r$. Очевидно, совокупность $c_1, c_2, \dots, c_n$ является решением исходной системы.

\textbf{Теорема.} Придавая свободным неизвестным произвольные значения и вычисляя значения главных неизвестных, из полученной системы можно получить все решения исходной системы.

\begin{proof} Пусть $(c_1, \dots, c_r, c_{r+1}, \dots, c_n)$ --- произвольное решение (\ref{eq1}). Возьмём числа $(c_{r+1}, \dots, c_n)$ в качестве свободных переменных $x_{r+1}, \dots, x_n$ и будем вычислять значения главных неизвестных из системы (\ref{eq2}). Так как $(c_1, \dots, c_r, c_{r+1}, \dots, c_n)$ --- решение (\ref{eq1}), то $(c_1, \dots, c_r)$ --- решение системы (\ref{eq2}). Так как система (\ref{eq2}) имеет единственное решение, то в качестве решения можем получить только $(c_1, \dots, c_r)$. 
\end{proof}

\textbf{Опр.} Система линейных алгебраических уравнений с нулевой правой частью называется \textbf{однородной}.

\textbf{Теорема.} СЛАУ с $n$ неизвестными имеет единственное решение $\iff rgB = rgA = n$.

\begin{proof} Если $rgA < n$, то среди неизвестных будет хотя бы одно свободное неизвестное. Тогда получим бесконечно много решений. 
\end{proof}

\textbf{Общее решение СЛАУ}
Решим полученную систему (\ref{eq1}) относительно главных неизвестных: $x_1 = f_1(x_{r+1}, \dots, x_n), ~ \dots, ~ x_r = f_r(x_{r+1}, \dots, x_n)$, где $f_1, \dots, f_r$ -- однозначно определённые функции. Эти соотношения при произвольных $x_{r+1}, \dots, x_n$ описывают множество всех решений исходной системы и называются \textbf{общим решением} системы.
 
\todo{сюда бы пример нахождения общего решения}
 
\bigbreak
\begin{itemize}
    \item Однородная СЛАУ $Ax = 0$ всегда совместна: имеет тривиальное решение $x = \theta$.
    \item Однородная система с $n$ неизвестными имеет нетривиальное решение $\iff rgA < n$.
    \item Однородная система $Ax = 0$ с квадратной матрицей $A$ имеет нетривиальное решение $\iff |A| = 0$.
\end{itemize}






% -------- source --------
\bigbreak
[\cite[page 104-109]{kim}]
