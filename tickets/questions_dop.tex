dop 1. Теорема Поста о полноте систем функций в алгебре логики.

dop 2. Графы,  деревья,  планарные графы;  их свойства.  Оценка числа деревьев.

dop 3. Логика 1-го порядка.  Выполнимость и общезначимость. Общая схема метода резолюций.

dop 4. Логическое  программирование. Декларативная семантика и операционная семантика;  соотношение между ними.  Стандартная стратегия выполнения логических программ.

dop 5. Коды Боуза-Чоудхури-Хоквингема: определение, алгоритмы кодирование и декодирования.

dop 6. Теорема Редфилда-Пойа (без доказательства). Примеры применения.

dop 7. Язык Python как мультипарадигмальный язык программирования. Python как интерпретатор. Объектная модель Python.

dop 8. Язык ассемблера как машиннозависимый язык низкого уровня. Организация ассемблерной программы, секции кода и данных (на примере ассемблера nasm или masm). Основные этапы подготовки к счёту ассемблерной программы: трансляция, редактирование внешних связей (компоновка), загрузка.

dop 9. Операционные системы. Управление оперативной памятью в вычислительной системе. Алгоритмы и методы организации и управления страничной оперативной памятью.

dop 10. Зависимости  в  реляционных  отношениях:  функциональные,  многозначные,  проекции/соединения. Проектирование реляционных БД на основе принципов нормализации отношений. Нормальные формы.

dop 11. Закон Амдала, его следствия. Граф алгоритма. Критический путь графа алгоритма, ярусно-параллельная форма графа алгоритма. Этапы решения задач на параллельных вычислительных системах.

dop 10. Глобальные и локальные модели освещения в компьютерной графике. Модель Фонга.

dop 11. Классификация  языков,  определяемых  конечными  автоматами,  регулярными  выражениями  и праволинейными грамматиками. Эквивалентность и минимизация конечных автоматов.

dop 12. Классификация языков, определяемых конечными автоматами, регулярными выражениями и праволинейными грамматиками. Эквивалентность и минимизация конечных автоматов. 

dop 13. Функции FIRST и FOLLOW. LL(l)-грамматики. Конструирование таблицы предсказывающего анализатора.

dop 14. Понятие имитационной модели. Примеры средств имитационного моделирования. Типовая архитектура средств имитационного моделирования (OmNet++, NS3).

dop 15. Алгоритм имитации отжига. Проблема возможности потери лучшего решения. Способы распараллеливания алгоритма имитации отжига.

dop 16. Основные  понятия  криптографии.  Односторонняя  функция  с  секретом.  Протокол  Диффи-Хеллмана выработкиобщего секретного ключа по открытому каналу связи.

dop 17. Основные  принципы  построения  и  архитектура  сети  Интернет.  Алгоритмы  и  протоколы  внешней  и внутренней маршрутизации. Явление перегрузки и методы борьбы с ней.

dop 18. Теоретические основы передачи данных, физический уровень стека протоколов. Системы передачи данных Ethernet и Wi-Fi: алгоритмы работы, управление множественным доступом к каналу.

dop 19. Базисные типы данных в языках программирования. Основные проблемы, связанные с базисными типами и способы их решения в различных языках. Понятие абстрактного типа данных и способы его реализации в современных языках программирования.

dop 20. Понятие  о  парадигме  программирования.  Основные  парадигмы  программирования.  Языки  и  парадигмы программирования.

dop 21. Основные  характеристики  функциональных  языков  программирования.  Использование  понятий функционального  программирования  (замыкания,  анонимные функции)  в  современных  объектно-ориентированных языках.

dop 22. Синхронизация  в  распределенных  системах.  Синхронизация  времени.  Логические  часы.  Выборы координатора. Взаимное исключение. Координация процессов.

dop 23. Отказоустойчивость  в  распределенных  системах.  Типы  отказов.  Фиксация  контрольных  точек  и восстановление после отказа. Репликация и протоколы голосования. Надежная групповая рассылка.

dop 24. Распределенные файловые системы. Доступ к директориям и файлам. Семантика одновременного доступа к одному файлу нескольких процессов. Кэширование и размножение файлов.

dop 25. Промежуточные  представления  программы:  абстрактное  синтаксическое  дерево;  последовательность трехадресных инструкций. Базовые блоки и граф потока управления.

dop 26. Локальная  оптимизация при  компиляции  программы. Ориентированный  ациклический  граф  и  метод нумерации значений.

dop 27. Глобальная оптимизация при компиляции программы. Построение передаточных функций базовых блоков. Монотонные и дистрибутивные передаточные функции. Метод неподвижной точки и его применение для нахождения достигающих определений.

dop 28. Постановка задачи дискретной оптимизации. Метод ветвей и границ. Задача целочисленного линейного программирования.

dop 29. Комбинаторные методы нахождения оптимального пути в графе.

dop 30. Потоки в сетях. Алгоритм построения максимального потока. Оценка сложности алгоритма.
