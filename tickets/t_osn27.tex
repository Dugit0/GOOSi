\subsection{OSN 27 Функции алгебры логики. Реализация их формулами. Совершенная дизъюнктивная нормальная форма.}

Пусть $E_{2}$ = {0,1} - основное множество. Тогда $E^n_{2} = \{(a_{1}, a_{2}, \ldots, a_{n}) \} \mid \forall i, a_{i} \in E_{2}\}$. Тогда всюду определенной булевой функцией назовем отображение $f_{n}(x_{1}, x_{2}, \ldots, x_{n})$: $E^n_{2} \rightarrow E_{2}$. Такую функцию можно задать таблично, а можно как суперпозицию других, более простых функций. Например, для n=1: \\

\begin{tabular}{ l | l l l l}
x & 0 & 1 & x & $\overline{x}$ \\
\hline
0 & 0 & 1 & 0 & 1\\
1 & 0 & 1 & 1 & 0\\
\end{tabular}


При этом функция 0 называется константным нулем, функция 1 - константной единицей, функция х - тождественной, а функция $\overline{x}$ - отрицанием х.  

Для n = 2: \\
\begin{tabular}{ l l | l l l l l l l}
x & y & $f_{1}$ & $f_{2}$ & $f_{3}$ & $f_{4}$ & $f_{5}$ & $f_{6}$ & $f_{7}$ \\
\hline
0 & 0 & 0 & 0 & 0 & 1 & 1 & 1 & 1 \\
0 & 1 & 1 & 0 & 1 & 1 & 0 & 1 & 0 \\
1 & 0 & 1 & 0 & 1 & 0 & 0 & 1 & 0 \\
1 & 1 & 1 & 1 & 0 & 1 & 1 & 0 & 0 \\
\end{tabular}

При этом: 
$f_{1}$ - дизъюнкция $f_{1} = x \vee y$\\
$f_{2}$ - конъюнкция $f_{2} = x \wedge y$ \\
$f_{3}$ - сложение по модулю два $f_{3} = x \oplus y$ \\
$f_{4}$ - импликация $f_{4} = x \rightarrow y$ \\
$f_{5}$ - эквивалентность $f_{5} = x \equiv y$\\
$f_{6}$ - штрих Шеффера $f_{6} = x \mid y$ \\
$f_{7}$ - стрелка Пирса $f_{7} = x \downarrow y$ \\

\textbf{Лемма (о числе слов)}: В алфавите $A = \{a_{1}, a_{2}, \ldots, a_{r}\}$ из $r$ букв можно построить ровно $r^m$ различных слов длины $m$. \\
\begin{proof}
Проведем индукцию по m. Для $m = 1$ утверждение очевидно. Пусть утверждение леммы верно для $m - 1$, то есть существует ровно $r^{(m-1)}$ различных слов длины $m - 1$. Для каждого такого слова длины $m-1$ существует ровно $r$ возможностей добавить одну букву в конец. Так как всего слов длины $m-1$ --- $r^{(m-1)}$, то различных слов длины m получится r*$r^{(m-1)}$ = $r^m$. 
\end{proof}

\faEye \ таблицу некоторой функции алгебры логики от n переменных:

\begin{tabular}{ l l l l l }
$x_{1}$ & $x_{2}$ & ... & $x_{n}$ & f \\
0 & 0 & ... & 0 & $a_{1}$ \\
0 & 0 & ... & 1 & $a_{2}$\\
... & ... & ... & ... & ...\\
1 & 1 & ... & 1 & $a_{2_{n} - 1}$\\
\end{tabular}

Для ее задания необходимо и достаточно определить ее значения на $2^{n}$ наборах. Так, получим, что всего различных функций от n переменных столько, сколько существует различных наборов из нулей и единиц длины $2^{n}$, то есть $2^{2^{n}}$. \\

Переменная $x_{i}$ называется \textbf{существенной} переменной функции алгебры логики $f_{n} (x_{1}, x_{2}, \ldots, x_{n})$, если существуют такие $(a_{1}, \ldots,a_{i-1}, a_{i+1}, \ldots, a_{n})$, что \\
    $f_{n} (a_{1},\ldots,a_{i-1}, 0, a_{i+1}, \ldots, a_{n}) \neq f_{n} (a_{1},\ldots ,a_{i-1}, 1, a_{i+1}, \ldots, a_{n}$) \\
Такие наборы, отличающиеся лишь одной переменной $x_{i}$, называются \textbf{соседними} по $x_{i}$. В противном случае переменная $x_{i}$ называется \textbf{фиктивной}. \\
Если $x_{i}$ - фиктивная переменная функции f, то функция f однозначно определяется некоторой функцией $g_{n} (x_{1}, \ldots ,x_{i-1}, x_{i+1}, \ldots, x_{n})$. Таблицу любой функции можно расширить путем введения новых фиктивных переменных. 
Две функции алгебры логики называются равными, если одну из них можно получить путем добавления и изъятия любого числа фиктивных переменных. 
Пусть имеется некоторое множество функций: \\
$A = \{f_{1}(\ldots), f_{2}(\ldots), \ldots, f_{n}(\ldots)\}$\\
Введем понятие \textbf{формулы} над А: 
\begin{enumerate}
    \item Любая функция из А называется формулой над А. 
    \item Если $f(x_{1}, x_{2}, ..., x_{n}) \in A$ и для любого $i H_{i}$ - либо переменная, либо формула над А, то выражение вида f($H_{1}$, $H_{2}$, ..., $H_{n}$) - формула над А. 
    \item Только те объекты называются формулами над А, которые можно построить с помощью пунктов 1 и 2. 
\end{enumerate}

Основные эквивалентности: 
\begin{itemize}
    \item Коммутативность: \\
            $x \vee y = y \vee x$ \\
            $yx = xy$ \\
            $x + y = y + x$ \\
            $x \equiv y = y \equiv x$
    \item Ассоциативность \\
            $(x \vee y) \vee z = x \vee (y \vee z) = x \vee y \vee z$ \\
            $xyz = (xy)z = x(yz)$ \\
            $x + y + z = (x + y) + z = x + (y + z)$
    \item Дистрибутивность \\
            $(x + y)\wedge z = (x\wedge z) + (y\wedge z)$ \\
            $(x \vee y)\wedge z = (x \wedge z)  \vee (y\wedge z)$ \\
            $(x\wedge y) \vee z = (x\vee z)\wedge(y \vee z)$
    \item Закон снятия двойного отрицания \\
            $\overline{\overline{x}} = x$
    \item Правила де Моргана: \\
            $\overline{x \vee y} = \overline{x}\wedge\overline{y}$ \\
            $\overline{x \wedge y} = \overline{x}\vee\overline{y}$
    \item Законы поглощения: \\
            $x \vee x = x$ \\
            $x \wedge x = x$ \\
            $x \vee \overline{x} = 1$ \\
            $x \wedge \overline{x} = 0$ \\
            $x \vee 1 = 1$ \\
            $x \wedge 1 = x$ \\
            $x \vee 0 = x$ \\
            $x \wedge 0 = 0$
    \item остальные формулы: \\
            $x\mid y = \overline{xy}$ \\
            $x \downarrow y = \overline{x \vee y}$ \\
            $x \rightarrow y = \overline{x} \vee y$ \\
            $x + y = (x \wedge \overline{y}) \vee (y \wedge \overline{x})$ \\
            $x \sim y = \overline{x + y} = (xy) \vee (\overline{x} \overline{y})$
\end{itemize}

Приоритет конъюнкции выше, чем приоритеты дизъюнкции и суммы по модулю два. 

\textbf{x в степени $\sigma$} называется функция $x^{\sigma} = x$, если  $\sigma = 1$; $x^{ \sigma} = \overline{x}, \sigma = 0$.

\bigbreak
\textbf{Теорема о разложении функции алгебры логики по переменным}: $\forall$ функции алгебры логики $f(x_{1}, x_{2},\ldots, x_{n})$ и $\forall k \in [1,n]$ справедливо следующее равенство:

$f(x_{1}, x_{2}, \ldots, x_{n}) = $

$ = \bigvee_{(\sigma_1,\ldots,\sigma_k)\in E^k}{x^{\sigma_{1}}_{1}  x^{\sigma_{2}}_{2}  \ldots  x^{\sigma_{k}}_{k}  f (\sigma_{1},\sigma_{2},\ldots,\sigma_{k},x_{k+1}, x_{k+2}, \ldots, x_{n})}$

\begin{proof}
$\forall$ набора $a = (a_{1}, \ldots, a_{n})$ вычислим значение правой части на этом наборе. Как только один из сомножителей будет равен 0, вся конъюнкция обратится в 0. Таким образом, из ненулевых конъюнкций останется лишь одна --- та,  в которой $a_{i} = \sigma_{i}$ и

$\bigvee a^{\sigma_{1}}_{1}  a^{\sigma_{2}}_{2}  \ldots  a^{\sigma_{n}}_{n}  f (\sigma_{1}, \sigma_{2},\ldots,\sigma_{k},a_{k+1}, a_{k+2}, \ldots, a_{n}) = 0 \vee 0 \vee \ldots \vee 0 \vee a^{a_{1}}_{1}  a^{a_{2}}_{2}  \ldots  a^{a_{n}}_{n}  f(a_{1}, a_{2},\ldots,a_{n})$ \\
Так как $x^{x} = 1$, указанное выражение равно $f(a_{1}, \ldots, a_{n}$).
\end{proof}

\textbf{Литерал} -- это переменная или отрицание переменной.

\textbf{Конъюнкция} -- любое произведение нескольких литералов, не содержащее одинаковых и противоположных литералов.

\textbf{Дизъюнктивная нормальная форма (ДНФ)} -- дизъюнкция нескольких различных конъюнкций (одна отдельная конъюнкция -- тоже ДНФ).

\textbf{Совершенная дизъюнктивная нормальная форма (СДНФ)} -- ДНФ, в каждой конъюнкции которой есть литерал каждой переменной.

\bigbreak
\textbf{Теорема о совершенной дизъюнктивной нормальной форме}: $\forall$ функции алгебры логики $f(x_{1}, x_{2}, \ldots, x_{n})$, отличной от тождественного нуля, справедливо следующее представление:

 $f(x_{1}, x_{2}, \ldots, x_{n}) = \bigvee x^{\sigma_{1}}_{1}   x^{\sigma_{2}}_{2}   \ldots   x^{\sigma_{n}}_{n}$
 
\begin{proof} 
Пусть $f(x_{1}, x_{2}, \ldots, x_{n})$ отлична от тождественного нуля. Разложим данную функция по $k = n$ переменным: \\
$f(x_{1}, x_{2}, \ldots, x_{n}) = \bigvee x^{\sigma_{1}}_{1}   x^{\sigma_{2}}_{2}   \ldots   x^{\sigma_{n}}_{n} f(\sigma_{1}, \sigma_{2},\ldots,\sigma_{n})$

при этом $(\sigma_{1}, \sigma_{2},\ldots,\sigma_{n}) \in E^n_{2}$.\\

Это может быть переписано в эквивалентном виде:

$f(x_1, x_2, \dots, x_n) = \bigvee_{(\sigma_1, \sigma_2, \dots, \sigma_n): f(\sigma_1, \sigma_2, \dots, \sigma_n) = 1} x_1^{\sigma_1}  \dots  x_n^{\sigma_n}$

Это представление называется \textbf{СДНФ}
\end{proof}
% $\bigvee x^{\sigma_{1}}_{1}   x^{\sigma_{2}}_{2}   \ldots   x^{\sigma_{n}}_{n} f(\sigma_{1}, \sigma_{2},\ldots,\sigma_{n}) \vee x^{\sigma_{1}}_{1}   x^{\sigma_{2}}_{2}   \ldots   x^{\sigma_{n}}_{n} f(\sigma_{1}, \sigma_{2},\ldots,\sigma_{n})$ \\

% при этом: $(\sigma_{1}, \sigma_{2},\ldots,\sigma_{n})$:$f(\sigma_{1}, \sigma_{2},\ldots,\sigma_{n}) = 1$ - выполнено для первого слагаемого, для второго - $(\sigma_{1},\sigma_{2},\ldots,\sigma_{n}):f(\sigma_{1}, \sigma_{2},\ldots,\sigma_{n}) = 0$. \\




% -------- source --------
\bigbreak
[\cite[page 69-96]{replace_me}]
