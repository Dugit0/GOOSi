\subsection{DOP 3 Логика 1-го порядка.  Выполнимость и общезначимость. Общая схема метода резолюций.}

Базовые символы: 
\begin{itemize}
    \item Предметные переменные $Var = \{x_1, x_2, \ldots, x_k, \ldots\}$;
    \item Предметные константы $Const = \{c_1, c_2,\ldots, c_l,\ldots\}$;
    \item Функциональные символы $Func = \{f^{n_1}_1,f^{n_2}_2,\ldots, f^{n_r}_r,\ldots\}$;
    \item Предикатные символы $Pred = \{P^{m_1}_1,P^{m_2}_2, \ldots, P^{m_s}_s,\ldots\}$.
\end{itemize}
Тройка $\langle Const,Pred,Func\rangle$ называется \textbf{сигнатурой алфавита}.

Логические связки и кванторы: 

Конъюнкция --- $\vee$ \\
Дизъюнкция --- $ \wedge $ \\
Отрицание --- $ \lnot $ \\
Импликация --- $ \longrightarrow $ \\
Квантор всеобщности --- $ \forall $ \\
Квантор существования --- $\exists$ \\


Определение терма: 
\textbf{Терм} --- это \\
$x$, если $x \in Var$, $x$ --- переменная;

$c$, если $c \in Const$, $c$ --- константа;

$f^n(t_1, t_2,\ldots, t_n)$,  если $f^n \in Func$, $t_1, t_2,\ldots, t_n$ --- термы, --- составной терм.

$Term$ --- множество термов заданного алфавита.
$Var_t$ --- множество переменных, входящих в состав терма t.

$t(x_1, x_2,\ldots, x_n)$ --- запись обозначающая терм t, у которого ${Var}_t \subseteq \{x_1, x_2,\ldots, x_n\}$.\\

\textbf{Формула} --- это либо атомарная формула
$P_m(t_1, t_2,\ldots, t_m)$ , если $ P_m \in Pred $, $\{t_1, t_2,\ldots, t_m\} \subseteq Term $; 
либо составная формула
$\bar{\phi}$, $\psi \vee \phi$ и т.д., если $\psi, \phi$ — формулы; 

Квантор связывает ту переменную, которая следует за ним.
Вхождение переменной в области действия квантора, \textbf{связывающего} эту переменную, называется связанным .
Вхождение переменной в формулу, не являющееся связанным, называется \textbf{свободным}.
Переменная называется \textbf{свободной}, если она имеет свободное вхождение в формулу.

$\psi(x_1, x_2,\ldots, x_n)$ — запись, обозначающая формулу $\psi$, у которой $ Var_{\psi} \supset \{x_1, x_2,..., x_n\}$.
Если ${Var}_{\psi}$ = 0, то формула $\psi$ называется
замкнутой формулой , или предложением .
$CForm$ — множество всех замкнутых формул.

\textbf{Интерпретация сигнатуры} --- это $\langle D_I, \overline{Const}, \overline{Func}, \overline{Pred} \rangle$, где
    \begin{itemize}
        \item $D_I$ --- непустое множество, которое называется областью интерпретации, предметной областью, универсумом.
        \item $\overline{Const}$ --- оценка констант --- сопоставляет каждой константе предмет из области интерпретации.
        \item $\overline{Func}$ --- оценка функциональных символов --- сопоставляет $n$-местной функции $f$ функцию из области интерпретации.
        \item $\overline{Pred}$ --- оценка предикатных символов --- сопоставляет каждому предикатному символу отношение на области интерпретации.
    \end{itemize}
    

\textbf{Отношение выполнимости} обозначается как $\models$.
Пусть $I$ --- интерпретация.
\newline $I \models \varphi(x_1 , \dots, x_n)[d_1 , \dots, d_n]$, т.е. формула выполнима в интерпретации $I$ на наборе $d_1, \dots, d_n$, если
\begin{enumerate}
    \item $\varphi(x_1, \dots, x_n) = P(t_1, \dots, t_m)$ и $ \overline{P} \left(t_1 [d_1 , \dots, d_n], \dots, t_n [d_1 , \dots, d_n]\right) = True $,
    \item $\varphi$ имеет вид $\psi_1 \wedge \psi_2$ и обе формулы выполнимы на наборе,
    \item $\varphi$ имеет вид $\psi_1 \vee \psi_2$ и хотя бы одна из формул выполнима на наборе,
    \item $\varphi$ имеет вид $\psi_1 \rightarrow \psi_2$ и на наборе либо выполнима $\psi_2$, либо невыполнима $\psi_1$,
    \item $\neg \varphi$ невыполнима на наборе,
    \item если $\varphi(x_1, \dots, x_n) = \exists x_0 \psi(x_0, x_1, \dots, x_n)$ и для некоторого элемента $d_0,~d_0 \in D_I$ имеет место $I \models \psi(x_0, \dots, x_n)[d_1, \dots, d_n]$,
    \item если $\varphi(x_1, \dots, x_n) = \forall x_0 \psi(x_0, x_1, \dots, x_n)$ и для любого элемента $d_0,~d_0 \in D_I$ имеет место $I \models \psi(x_0, \dots, x_n)[d_1, \dots, d_n]$.
\end{enumerate}

\textbf{Формула выполнима в интерпретации}, если существует хотя бы один набор элементов интерпретации, на котором формула выполнима.

\textbf{Формула истинна} в интерпретации, если она выполнима на любом наборе элементов интерпретации.

\textbf{Формула выполнима}, если существует интерпретация, в которой она выполнима.

\textbf{Формула общезначима}, если она истинна в любой интерпретации.

Формула без свободных переменных называется \textbf{замкнутой формулой} (предложением).

Пусть $\Gamma$ — некоторое множество замкнутых формул, $\Gamma \subseteq CForm$. Тогда каждая интерпретация I, в которой выполняются все формулы множества $\Gamma$, называется \textbf{моделью для множества} $\Gamma$. \\

Пусть $\Gamma$ — некоторое множество замкнутых формул, и $\psi$ --- замкнутая формула. Формула $\psi$ называется логическим следствием множества предложений (базы знаний) $\Gamma$, если каждая модель для множества формул $\Gamma$ является моделью для формулы $\psi$, т. е. для любой интерпретации I верно $I \models \Gamma \Longleftrightarrow I \models \psi$ \\
Запись $\Gamma \models \psi$ обозначает, что $\psi$ --- логическое следствие $\Gamma$.

\textbf{Теорема о логическом следствии}.
Пусть $\Gamma = \{\psi_1,\ldots,\psi_n\} \subseteq CForm $, $\phi \in CForm $. Тогда $\Gamma \models \phi \iff  \models \psi_1 \wedge \ldots \wedge \psi_{n} \longrightarrow \phi$.

\begin{proof}
$(\Rightarrow)$ Пусть $I$ — произвольная интерпретация. 
Если $I \nvDash \psi_1 \wedge \ldots \wedge \psi_{n}$, то $I \ldots \psi_1 \wedge \ldots \wedge \psi_{n} \longrightarrow \phi$. \\
Если $I \models \psi_1 \wedge \ldots \wedge \psi_{n}$, то $I \models \psi_i$, $i \in [1,n]$, т. е. $I$ --- модель для $\Gamma$. Поскольку $\Gamma \models \phi$, получаем $I \models \phi$. \\
Значит, $I \models \psi_1 \wedge \ldots \wedge \psi_{n} \longrightarrow \phi$.
Таким образом, для любой интерпретации $I$ имеет место $I \models \psi_1 \wedge \ldots \wedge \psi_{n} \longrightarrow \phi$.
Значит, $\psi_1 \wedge \ldots \wedge \psi_{n} \longrightarrow \phi$ --- общезначимая формула.

$(\Leftarrow)$ Пусть $I$ — модель для множества предложений $\Gamma$, т. е. $I \models \psi_i$, $i \in [1,n]$. Тогда $I \models \psi_1 \wedge \ldots \wedge \psi_{n}$. Поскольку $\psi_1 \wedge \ldots \wedge \psi_{n} \longrightarrow \phi$ - общезначимая формула, имеет место $I \models \psi_1 \wedge \ldots \wedge \psi_{n} \longrightarrow \phi$. Значит $I \models \phi$. Так как $I$ --- произвольная модель для $\Gamma$, приходим к заключению $\Gamma \models \phi$.
\end{proof}

\textbf{Общезначимые формулы} --- это каналы причинно-следственной связи, по которым передаются знания, представленные в виде логических формул, преобразуясь при этом из одной формы в другую.
Практически важно уметь определять эти каналы и настраивать их на извлечение нужных знаний.
\begin{itemize}
    \item База знаний --- множество предложений $\Gamma$;
    \item Запрос к базе знаний --- предложение $\phi$;
    \item Получение ответа на запрос --- проверка логического следствия $\Gamma \models \phi$.
\end{itemize}

\textbf{Метод резолюции}

Подстановка $\theta$ -- \textbf{унификатор} выражений $E_1, E_2$, если $E_1 \theta = E_2 \theta$. Подстановка $\theta$ -- \textbf{наиболее общий унификатор (НОУ)} выражений $E_1, E_2$, если $\theta$ -- унификатор выражений $E_1, E_2$ и для любого унификатора $\eta$ существует такая подстановка $\rho$, для которой верно $\eta = \theta \rho$.

Метод резолюций --- метод проверки формулы $\varphi$ на общезначимость. 
Общая схема метода:
\begin{enumerate}
    \item Свести проблему общезначимости к проблеме противоречивости: если формула общезначима, то ее отрицание невыполнимо: $\psi = \neg \varphi$
    \item Построить предварённую нормальную форму (ПНФ): $\psi = Q_1x_1 \dots Q_nx_n(D_1 \wedge \dots \wedge D_N)$, где $Q_i$ либо квантор существования, либо квантор всеобщности, а $D_i$ --- дизъюнкт из КНФ.
    \item Построить сколемовскую стандартную форму (ССФ): $\forall x_{i_1} \dots \forall x_{i_k}(D_1 \wedge \dots \wedge D_N)$.
    При этом формулы будут неэквивалентны, однако свойство противоречивости сохранится.
    \item Построить систему дизъюнктов: $S_\varphi \{D_1, \dots, D_N\}$
    \item Резолютивно (с использованием правил резолюции и склейки) вывести из системы $S_\varphi$ пустой дизъюнкт, тем самым доказав противоречивость.
    \begin{itemize}
        \item Правило резолюции: $\frac{D'_1 \vee L_1,~D'_2 \vee \neg L_2}{(D'_1 \vee D'_2)\theta},~\theta \in \text{НОУ}(L_1, L_2)$
        \item Правило склейки: $\frac{D'_1 \vee L_1 \vee L_2}{(D'_1 \vee L_1)\eta},~\eta \in \text{НОУ}(L_1,L_2)$
    \end{itemize}
\end{enumerate}


% -------- source --------
\bigbreak
[\cite[page 69-96]{replace_me}]