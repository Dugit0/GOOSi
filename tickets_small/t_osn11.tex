\subsection*{OSN 11 Алгебраические линии и поверхности второго порядка, канонические уравнения,  классификация.}

Если в пространстве $V_3$ зафиксированы точка $O$ и базис $\{e_1, e_2, e_3\}$, то говорят что в пространстве задана \textbf{афинная система координат} (или \textbf{общая декартова система координат}) $\{O, e_1, e_2, e_3\}$. Точка $O$ называется \textbf{началом координат}. Оси, проходящие через начало координат и определенные векторами $\{e_1, e_2, e_3\}$, называются \textbf{осями координат}. (Обозначается как $O_{xyz}$). Если вектора $e_i$ взаимно перепендикулярны, то задана \textbf{прямоугольная система координат}.

\bigbreak

Пусть $Oxy$ --- афинная система координат на плоскости. \textbf{Алгебраическая линия второго порядка} определяется уравнением $F(x, y) = 0$, где $F(x, y)$ --- алгебраический многочлен второй степени от переменных $x$ и $y$ с вещественными коэффициентами:
$$F(x,y)=a_{11}x^2 +2a_{12}xy+a_{22}y^2 +2a_{13}x+2a_{23}y+a_{33} =0,$$
$$~a_{11}^2 + a_{21}^2 + a_{22}^2\neq 0.$$
Это ур-е назывется \textbf{общим уравнением алгебраической линии второго порядка на плоскости}. Группа слагаемых $a_{11}x^2 + 2a_{12}xy + a_{22}y^2$ называется \textbf{квадратичной частью уравнения}, группа слагаемых $2a_{13}x + 2a_{23}y$ --- \textbf{линейной частью}, а $a_{33}$ --- свободным членом.

Введем обозначения:
$$A = \begin{pmatrix} a_{11} & a_{12} \\a_{12} & a_{22}\end{pmatrix},~ b=\begin{pmatrix} a_{13} \\ a_{23} \end{pmatrix},~ X=\begin{pmatrix}x \\ y\end{pmatrix},$$
$$B = \begin{pmatrix} a_{11} & a_{12} & a_{13} \\a_{12} & a_{22} & a_{23} \\ a_{13} & a_{23} & a_{33}\end{pmatrix} = \begin{pmatrix} A & b \\ b^T & a_{33}\end{pmatrix}$$

Тогда уравнение примет вид:
$$F(x,y) = X^TAX+2b^TX+a_{33}=0,~A=A^T,~A\neq\mathcal{O}.$$

\textbf{Теорема.} Общее уравнение линии второго порядка, заданное в прямоугольной декартовой системе координат, переходом к другой прямоугольной системе координат приводится к одному из следующих типов уравнений:
\begin{enumerate}
    \item $\lambda_1 x^2 + \lambda_2 y^2 + a_0 = 0$, где $\lambda_1\lambda_2\neq 0$
    \item $\lambda_2 y^2 + 2b_0x = 0$, где $\lambda_2 b_0\neq 0$
    \item $\lambda_2 y^2 + c_0 = 0$, где $\lambda_2\neq 0$
\end{enumerate}
Эти уранения называются \textbf{приведенными уравнениями} линии второго порядка.

\begin{proof}
\textbf{Шаг 1}: (преобразование базиса). \textbf{Метод вращений}. Если $a_{12} \neq 0$, то поворотом осей можно привести квадратичную часть $F(x,y)$ к сумме квадратов: $F(x,y) = a'_{11}x'^{2} + a'_{22}y'^{2} + a'_{13}x' + a'_{23}y' + a_{33} = 0$.

\textbf{Шаг 2}: (перенос начала). Если в полученном ур-е содержится ненулевой квадрат какой-либо переменной, то переносом начала можно освободиться от этой переменной в первой степени. Если $a'_{11} \neq 0$ и $a'_{22} \neq 0$, то 
$$
x'' = x' + \frac{a'_{13}}{a'_{11}}, \ y'' = y' + \frac{a'_{23}}{a'_{22}}, \ a'_{33} = a_{33} - \frac{{a'_{13}}^2}{a'_{11}} - \frac{{a'_{23}}^2}{a'_{22}}
$$

$$
a'_{11}x''^{2} + a'_{22}y''^{2} + a'_{33} = 0
$$
Все промежуточные и окончательные системы координат оставались прямоугольными, т.к. преобразования базиса с помощью ортогональной матрицы перехода сохраняют свойства ортонормированности.
\end{proof}


\textbf{Классификация ЛИНИЙ второго порядка}

\textbf{Теорема.} Общее уравнение линии второго порядка, заданное в прямоугольной декартовой системе координат, определяет одну и только одну из девяти линий. Для каждой из них существует прямоугольная система координат, в которой уравнение этой линии имеет \textbf{канонический вид}:

\textbf{I тип:}
\begin{enumerate}
    \item $\frac{x^2}{a^2} + \frac{y^2}{b^2} = \pm1$ --- эллипс (мнимый эллипс);
    \item $\frac{x^2}{a^2} \pm \frac{y^2}{b^2} = 0$ --- пара мнимых пересекающихся прямых (пара пересекающихся прямых); Только начало координат удовлетворяет ур-ю мним. пер. прям.
    \item $\frac{x^2}{a^2} - \frac{y^2}{b^2} = 1$ --- гипербола;
\end{enumerate}

\textbf{II тип:} $y^2 =2px,~p>0$ --- парабола;

\textbf{III тип:}
\begin{enumerate}
    \item $y^2 = \pm a^2,~a\neq 0$ --- пара параллельных прямых (пара мнимых параллельных прямых); Ни одна точна не удовлетворяет ур-ю мним. парал. прям.
    \item $y^2 = 0$ --- пара совпадающих прямых.
\end{enumerate}

\bigbreak

\textbf{Классификация ПОВЕРХНОСТЕЙ второго порядка}

Под \textbf{общим уравнением алгебраической поверхности} второго порядка в системе координат $Oxyz$ пространства понимают уравнение вида:

$F(x,y) = a_{11}x^2+a_{22}y^2+a_{33}z^2+2a_{12}xy+2a_{13}xz+2a_{23}yz+2b_1x+2b_2y+2b_3z+c=0$,
где не все коэффициенты $a_{ij}$ равны нулю, $a_{ij} = a_{ji}$

Введем обозначения:
$$A = \begin{pmatrix} a_{11} & a_{12} & a_{13} \\a_{12} & a_{22} & a_{23} \\ a_{13} & a_{23} & a_{33}\end{pmatrix},~ b=\begin{pmatrix} b_1 \\ b_2 \\ b_3 \end{pmatrix},~ X=\begin{pmatrix}x \\ y \\ z\end{pmatrix},
$$

$$~B = \begin{pmatrix} a_{11} & a_{12} & a_{13} & b_1 \\a_{12} & a_{22} & a_{23} & b_2 \\ a_{13} & a_{23} & a_{33} & b_3 \\ b_1 & b_2 & b_3 & c \end{pmatrix} = \begin{pmatrix} A & b \\ b^T & c\end{pmatrix}$$

Тогда уравнение примет вид:
$$F(x, y) = X^TAX+2b^TX+c=0,~A\neq\mathcal{O},~A=A^T.$$

\textbf{Теорема.} С помощью ортогонального преобразования координат (т.е. простым вращением и простым отражением) и параллельного перноса уравнение можно привести к одному из следующих типов:
\begin{enumerate}
    \item $\lambda_1x^2 + \lambda_2y^2 +\lambda_3z^2 + a_0 = 0, \ \lambda_1 \lambda_2 \lambda_3 \neq 0$
    \item $\lambda_1x^2 + \lambda_2y^2 + b_0z = 0, \ \lambda_1 \lambda_2 b_0 \neq 0$
    \item $\lambda_1x^2 + \lambda_2y^2 + c_0 = 0, \ \lambda_1 \lambda_2 \neq 0$
    \item $\lambda_2y^2 + p_0x = 0 , \ \lambda_1 p_o \neq 0$
    \item $\lambda_2y^2 + q = 0, \ \lambda_1 \neq 0$
\end{enumerate}


\textbf{Теорема.} Для любой алгебраической поверхности второго порядка существует прямоугольная декартова система координат, в которой уравнение этой поверхности имеет канонический вид:

\textbf{I тип:}
\begin{enumerate}
    \item $\frac{x^2}{a^2} + \frac{y^2}{b^2} + \frac{z^2}{c^2} = \pm1$ --- эллипсоид (мнимый эллипсоид); Ни одна точка пространства не удовлетворяет ур-ю мним. эллипс.
    \item $\frac{x^2}{a^2} + \frac{y^2}{b^2} + \frac{z^2}{c^2} = 0$ --- вырожденный эллипсоид; Удовлетворяет только начало координат.
    \item $\frac{x^2}{a^2} + \frac{y^2}{b^2} - \frac{z^2}{c^2} = \pm1$ --- однополостный гиперболоид (двухполостный гиперболоид);
    \item $\frac{x^2}{a^2} + \frac{y^2}{b^2} - \frac{z^2}{c^2} = 0$ --- конус;
\end{enumerate}

\textbf{II тип:}
 $ 2 Z = \frac{x^2}{a^2} \pm \frac{y^2}{b^2}$ --- эллиптический параболоид (гиперболический параболоид);

\textbf{III тип:}
\begin{enumerate}
    \item $\frac{x^2}{a^2} + \frac{y^2}{b^2} = \pm1$ --- эллиптический цилиндр (мнимый эллиптический цилиндр); Ни одна точка пространства не удовлетворяет ур-ю мним. эл. цил.
    \item $\frac{x^2}{a^2} - \frac{y^2}{b^2} = 1$ --- гиперболический цилиндр;
    \item $\frac{x^2}{a^2} + \frac{y^2}{b^2} = 0$ --- пара мнимых пересекающихся плоскостей;
    \item $\frac{x^2}{a^2} - \frac{y^2}{b^2} = 0$ --- пара пересекающихся плоскостей;
\end{enumerate}
\textbf{IV тип:} $y^2 = 2px,~p > 0$ --- параболический цилиндр;

\textbf{V тип:}
\begin{enumerate}
    \item $y^2 = \pm a^2$ --- пара параллельных плоскостей (пара мнимых параллельных плоскостей);
    \item $y^2 = 0$ --- пара совпадающих плоскостей.
\end{enumerate}


% -------- source --------
\bigbreak
[\cite[page 192-200, 329-341]{kim}]
