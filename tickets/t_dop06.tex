\subsection*{DOP 6 Теорема Редфилда-Пойа (без доказательства). Примеры применения.}

Пусть даны:
\begin{itemize}
    \item группа $\left< G, \circ, e \right>, \space |G| = n$
    \item множество $T, \space |T| = N>0$
    \item $B_{ij}(T)$ - множество всех перестановок элементов $T$ (биекций на $T$).
    \item $S_{T}$ - симметрическая группа множества $T$: $S_{T} = \left< B_{ij}(T), *, 1_{T} \right>$.
\end{itemize}

\textbf{Действием $\alpha$ группы $G$ на множестве $T$} называется гомоморфизм из группы $G$ в группу $S_{T}$.

\textbf{Отношением эквивалентности} $\sim_{g}$ на $T$ \underline{для перестановки} $g$ называется $t \sim_{g} t' \Leftrightarrow \exists k \in \mathbb{Z}: g^k(t) = t'$.

\textbf{$g$-циклами} называются смежные классы эквивалентности $\sim_{g}$. Элементы этих классов образуют циклы:
$t \overset{g}{\to} t' \overset{g}{\to} \ldots \overset{g}{\to} t$.

\textbf{Типом перестановки} называется $Type(g) = \left< v_{1}, v_{2}, \dots, v_{N} \right>$,
где $v_{1}, v_{2}, \dots, v_{N}$ -- количества циклов длины $1, 2, \dots, N$.

Число всех $g$-циклов равняется $C(g) = \sum\limits_{k=1}^{N} v_{k}(g)$.

Пример: $T = \{1, \dots, 10\}$ и
$$g = \begin{pmatrix} 1&2&3&4&5&6&7&8&9&10 \\ 9&6&1&8&5&2&7&10&3&4 \end{pmatrix} = \
(1,9,3)(2,6)(4,8,10)(5)(7) = (2,6)(1,9,3)(4,8,10).$$
Тогда $Type(g) = \left< 2, 1, 2, 0, 0, 0, 0, 0, 0, 0 \right>$, $C(g) = 5$, $N = 10$.\\

\textbf{Весом $w(g)$ перестановки} $g \in G$ называется $w(g) = x_{1}^{v_{1}} \cdot \dots x_{N}^{v_{N}}$, где $\left< v_{1}, v_{2}, \dots, v_{N} \right> = Type(g)$.\\

\textbf{Цикловым индексом} $Z(G \underset{\alpha}{:}T, x_{1}, \dots, x_{N})$ действия группы $G$ на множестве $T$ называют средний вес подстановок в группе: $Z(G \underset{\alpha}{:}T, x_{1}, \dots, x_{N}) = \frac{1}{|G|} \sum\limits_{g \in G} w(g) = \frac{1}{|G|} \sum\limits_{g \in G} x_{1}^{v_{1}} \cdot \dots x_{N}^{v_{N}}$.\\


\textbf{Теорема Пойа}

Пусть заданы множество $T$, группа $G$ и действие $G \underset{\alpha}{:}T$.
\begin{enumerate}
    \item Припишем каждому элементу из $T$ одно из $r$ значений (неформально: покрасим в один из $r$ цветов). Всего, очевидно, имеется $r^{N}$ раскрасок.
    \item Не будем различать раскраски, если элементы $t$ и $t' = g(t)$ раскрашены одинаково.
\end{enumerate}
Тогда число неэквивалентных раскрасок равно числу классов эквивалентности и вычисляется по формуле $C(G \underset{\alpha}{:}T) = \left . Z(G \underset{\alpha}{:}T, x_{1}, \dots, x_{N}) \right|_{x_{1} = \ldots = x_{N} = r}$.

\textbf{Пример. Задача о квадратах $2 \times 2$.}

\textbf{Условие:}\\
Сколькими способами можно раскрасить доску $2 \times 2$ в $r$ цветов, если раскраски, переходящие друг в друга при вращении квадрата, считаются одинаковыми?\\

\textbf{Решение:}\\
Рассмотрим группу вращений квадрата $\mathbb{Z}_{4} = \{e, t, t^{2}, t^{3}\}$, где $e$ -- вращение квадрата на $0\degree$, а $t$ -- вращение на $90\degree$ (очевидно, что $t^4 = e$).

Рассмотрим квадрат $A = \begin{bmatrix} 1 & 2 \\ 4 & 3 \end{bmatrix}$ (это не матрица, а квадрат в котором 4 ячейки под номерами 1, 2, 3 и 4!).

$eA = \begin{bmatrix} 1 & 2 \\ 4 & 3 \end{bmatrix}$,
$tA = \begin{bmatrix} 4 & 1 \\ 3 & 2 \end{bmatrix}$,
$t^{2}A = \begin{bmatrix} 3 & 4 \\ 2 & 1 \end{bmatrix}$,
$t^{3}A = \begin{bmatrix} 2 & 3 \\ 1 & 4 \end{bmatrix}$\\

Таким образом\\
$e = (1)(2)(3)(4), \space Type(e) = \left< 4, 0, 0, 0\right>, \space w(e) = x_{1}^{4}$

$t = (1432), \space Type(e) = \left< 0, 0, 0, 1\right>, \space w(e) = x_{4}^{1}$

$t^{2} = (13)(24), \space Type(e) = \left< 0, 2, 0, 0\right>, \space w(e) = x_{2}^{2}$

$t^{3} = (1234), \space Type(e) = \left< 0, 0, 0, 1\right>, \space w(e) = x_{4}^{1}$\\

Итого, цикловой индекс равен $Z = \frac{1}{|G|} (x_{1}^{4} + 2x_{4}^{1} + x_{2}^{2})$, подставляя $|G| = 4$ и $x_{1} = x_{2} = x_{4} = r$, $Z(r) = \frac{r^{4}+2r+r^{2}}{4}$.

\textbf{Ответ:} $Z(r) = \frac{r^{4}+2r+r^{2}}{4}$.


\textbf{Пример. Задача об ожерельях $N = 5, r = 3$.}

\textbf{Условие:}\\
Сколько разных ожерелий можно составить из 5 бусин 3 цветов?

\textbf{Решение:}\\
Рассмотрим группу вращений таких ожерелий $\mathbb{Z}_{5} = \{e, t, t^{2}, t^{3}, t^{4}\}$, где $e$ -- тождественная подстановка, а $t$ - вращение на одну бусину (очевидно, что $t^5 = e$).

$e = (1)(2)(3)(4)(5), \space Type(e) = \left< 5, 0, 0, 0, 0\right>, \space w(e) = x_{1}^{5}$

$t = (12345), \space Type(t) = \left< 0, 0, 0, 0, 1\right>, \space w(t) = x_{5}$

$t^2 = (13524), \space Type(t) = \left< 0, 0, 0, 0, 1\right>, \space w(t^2) = x_{5}$

$t^3 = (14253), \space Type(t) = \left< 0, 0, 0, 0, 1\right>, \space w(t^3) = x_{5}$

$t^4 = (15432), \space Type(t) = \left< 0, 0, 0, 0, 1\right>, \space w(t^4) = x_{5}$

Итого, цикловой индекс равен $Z = \frac{1}{|G|} (x_{1}^{5} + 4x_{5})$, подставляя $|G| = 5$ и $x_{1} = x_{5} = 3$, $Z(3) = \frac{1}{5} (3^5 + 4 \cdot 3) = 51$.

\textbf{Ответ:} 51.


% -------- source --------
\bigbreak
[\cite[page 132-161]{gurov_23}]
